  \title{Analysis of Tetris Ballistic Deposition and the Robustness of the KPZ Universality Class}% {{{
% \subtitle
% {Research Plan} % (optional)

\author[Le Chen]{Le Chen\\
  Auburn University\\[2em]
  \footnotesize Acknwolegement\\[0.5em]
  \textit{NSF 2246850, NSF 2443823, \& Simons Foundation Travel Grant (2022-2027)}\\[1.3em]
  \href{https://chenle02.github.io/2025-10-28_Emerging_Synergies_Banff_Le/}{\texttt{Talk available at: github.com/chenle02}}%
% \small University of Cardiff \\[1.5em]
% https://chenle02.github.io/2025-10-28_Emerging_Synergies_Banff_Le/
}
% - Use the \inst{?} command only if the authors have different
%   affiliation.
\institute[Auburn University]
{%
% \pgfuseimage{Auburn}
% Jointwork with\\[1em]
% Raluca Balan, Xia Chen and Nicholas Eisenberg \\
 % \vspace{2cm}
 }
% - Use the \inst command only if there are several affiliations.
% - Keep it simple, no one is interested in your street address.

% \date[Talk at Karlsruhe] % (optional)
% {\today }
\date[Banff]{
{\small Emerging Synergies between Stochastic Analysis and Statistical Mechanics} \\
{\small Banff, Alberta, Canada} \\
{\small October 28, 2025}
}

% This is only inserted into the PDF information catalog. Can be left
% out.

% If you have a file called "university-logo-filename.xxx", where xxx
% is a graphic format that can be processed by latex or pdflatex,
% resp., then you can add a logo as follows:

% \pgfdeclareimage[height=2cm]{Auburn}{figs/AU-Black.jpeg}

% Delete this, if you do not want the table of contents to pop up at
% the beginning of each subsection:
% \AtBeginSubsection[]
% {
%   \begin{frame}<beamer>{Outline}
%     \tableofcontents[currentsection,currentsubsection]
%   \end{frame}
% }

% If you wish to uncover everything in a step-wise fashion, uncomment
% the following command:

% \beamerdefaultoverlayspecification{<+->}
