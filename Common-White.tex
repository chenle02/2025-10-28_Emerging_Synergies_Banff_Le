% \usepackage{dbt}
\setbeamercovered{dynamic}
\usepackage{etex}
\usepackage{graphicx,url,psfrag}
% Safely include graphics: if missing, show a placeholder frame instead of erroring
\makeatletter
\let\OldIncludeGraphics\includegraphics
\renewcommand{\includegraphics}[2][]{%
  \IfFileExists{#2}{\OldIncludeGraphics[#1]{#2}}{\fbox{\scriptsize Missing: \texttt{#2}}}%
}
\makeatother
\usepackage{tikz}
\usetikzlibrary{
  decorations.pathreplacing,
  calc,
  decorations.fractals,
  through,
  shapes,
  patterns,
  arrows.meta,
  arrows,
  calligraphy,
  shapes.symbols,
  shapes.geometric,
  matrix,
  fit,
  shapes.geometric,
  intersections,
  positioning,
}
\usepackage{tikz-feynman}
\makeatletter
\tikzfeynmanset{compat=\tikzfeynman@version@major.\tikzfeynman@version@minor.\tikzfeynman@version@patch}
\makeatother
\usepackage{setspace}
% \usepackage[center]{subfigure}
\usepackage{enumerate}
\usepackage[makeroom]{cancel}
\usepackage{mathtools}
\usepackage{graphbox}
\usepackage{amssymb}
\usepackage{amsfonts}
\usepackage{mathrsfs}
\usepackage{comment}
\usepackage{xfrac}
\usepackage{pgfplots}
\usepackage{linguex}
\usepackage{ocgx2}
\usepackage{animate}
\excludecomment{codes}
% \usepackage{movie15}
% \usepackage[showframe]{geometry}
% \usepackage{enumitem}

%
% for warning sign
%
\usepackage{pgfplots}
\usepackage{stackengine}
\usepackage{scalerel}
\usepackage{xcolor}
\newcommand\dangersign[1][2ex]{%
  \renewcommand\stacktype{L}%
  \scaleto{\stackon[1.3pt]{\color{red}$\triangle$}{\tiny !}}{#1}%
}
% %  The following is to show codes:
\usepackage{listings}
% \usepackage{color}

\definecolor{dkgreen}{rgb}{0,0.6,0}
\definecolor{gray}{rgb}{0.5,0.5,0.5}
\definecolor{mauve}{rgb}{0.58,0,0.82}

\lstset{frame=tb,
  language=Java,
  aboveskip=3mm,
  belowskip=3mm,
  showstringspaces=false,
  columns=flexible,
  basicstyle={\small\ttfamily},
  numbers=none,
  numberstyle=\tiny\color{gray},
  keywordstyle=\color{blue},
  commentstyle=\color{dkgreen},
  stringstyle=\color{mauve},
  breaklines=true,
  breakatwhitespace=true,
  tabsize=3
}
\lstset{language=Python}

\lstset{ %
  language=Python,                     % the language of the code
  basicstyle=\footnotesize,       % the size of the fonts that are used for the code
  numbers=left,                   % where to put the line-numbers
  numberstyle=\tiny\color{gray},  % the style that is used for the line-numbers
  stepnumber=1,                   % the step between two line-numbers. If it's 1, each line
                                  % will be numbered
  numbersep=5pt,                  % how far the line-numbers are from the code
  backgroundcolor=\color{white},  % choose the background color. You must add \usepackage{color}
  showspaces=false,               % show spaces adding particular underscores
  showstringspaces=false,         % underline spaces within strings
  showtabs=false,                 % show tabs within strings adding particular underscores
  frame=single,                   % adds a frame around the code
  rulecolor=\color{black},        % if not set, the frame-color may be changed on line-breaks within not-black text (e.g. commens (green here))
  tabsize=2,                      % sets default tabsize to 2 spaces
  captionpos=b,                   % sets the caption-position to bottom
  breaklines=true,                % sets automatic line breaking
  breakatwhitespace=false,        % sets if automatic breaks should only happen at whitespace
  title=\lstname,                 % show the filename of files included with \lstinputlisting;
                                  % also try caption instead of title
  keywordstyle=\color{blue},      % keyword style
  commentstyle=\color{dkgreen},   % comment style
  stringstyle=\color{mauve},      % string literal style
  escapeinside={\%*}{*)},         % if you want to add a comment within your code
  morekeywords={*,...}            % if you want to add more keywords to the set
}
% \usepackage[usenames,dvipsnames]{color}
% \lstset{
%   language=R,                     % the language of the code
%   basicstyle=\tiny\ttfamily, % the size of the fonts that are used for the code
%   numbers=left,                   % where to put the line-numbers
%   numberstyle=\tiny\color{Blue},  % the style that is used for the line-numbers
%   stepnumber=1,                   % the step between two line-numbers. If it is 1, each line
%                                   % will be numbered
%   numbersep=5pt,                  % how far the line-numbers are from the code
%   backgroundcolor=\color{white},  % choose the background color. You must add \usepackage{color}
%   showspaces=false,               % show spaces adding particular underscores
%   showstringspaces=false,         % underline spaces within strings
%   showtabs=false,                 % show tabs within strings adding particular underscores
%   frame=single,                   % adds a frame around the code
%   rulecolor=\color{black},        % if not set, the frame-color may be changed on line-breaks within not-black text (e.g. commens (green here))
%   tabsize=2,                      % sets default tabsize to 2 spaces
%   captionpos=b,                   % sets the caption-position to bottom
%   breaklines=true,                % sets automatic line breaking
%   breakatwhitespace=false,        % sets if automatic breaks should only happen at whitespace
%   keywordstyle=\color{RoyalBlue},      % keyword style
%   commentstyle=\color{YellowGreen},   % comment style
%   stringstyle=\color{ForestGreen}      % string literal style
% }

% \usepackage[dvipsnames]{xcolor}
% \newcommand{\Cross}{\mathbin{\tikz [x=1.4ex,y=1.4ex,line width=.2ex] \draw (0,0) -- (1,1) (0,1) -- (1,0);}}%
% \beamerdefaultoverlayspecification{<+-| alert@+>} %(this will show line by line)
\beamerdefaultoverlayspecification{<+->} %(this will show line by

% \usepackage{natbib}
% \input{../myMathSymbols.tex}
% \newcommand{\tlMr}[4]{\:{}^{\hspace{0.2em}#1}_{#2} \hspace{-0.1em}#3_{#4}}

% Smiley face\Smiley{} \Frowny{}
\usepackage{marvosym}
% -------------------------------------------------
%  Set directory for figs
% -------------------------------------------------
\usepackage{grffile}
\graphicspath{{Codes/}}
% -------------------------------------------------
%  Define colors
% -------------------------------------------------
\def\refcolor{cyan}
\def\excolor{brown}
% \usepackage{color}
% \usepackage[dvipsnames]{xcolor}


% % % Define danger sign
\newcommand*{\TakeFourierOrnament}[1]{{%
\fontencoding{U}\fontfamily{futs}\selectfont\char#1}}
\newcommand*{\danger}{\TakeFourierOrnament{66}}


% -------------------------------------------------
%  Define short-hand symbols.
% -------------------------------------------------
\DeclareMathOperator{\Lip}{\mathit{L}}
\DeclareMathOperator{\LIP}{Lip}
\newcommand{\dom}{\mathcal{O}}
\newcommand{\E}{\mathbb{E}}
\newcommand{\D}{\mathbb{D}}
\newcommand{\ud}{\ensuremath{\mathrm{d}}}
\newcommand{\Ceil}[1]{\left\lceil #1 \right\rceil}
\newcommand{\Floor}[1]{\left\lfloor #1 \right\rfloor}
\newcommand{\sgn}{\text{sgn}}
\newcommand{\Lad}{\text{L}_{\text{ad}}^2}
\newcommand{\SI}[1]{\mathcal{I}\left[#1 \right]}
\newcommand{\SIB}[2]{\mathcal{I}_{#2}\left[#1 \right]}
\newcommand{\Indt}[1]{1_{\left\{#1 \right\}}}
\newcommand{\LadInPrd}[1]{\left\langle #1 \right\rangle_{\text{L}_\text{ad}^2}}
\newcommand{\LadNorm}[1]{\left|\left|  #1 \right|\right|_{\text{L}_\text{ad}^2}}
\newcommand{\Norm}[1]{\left|\left|  #1   \right|\right|}
\newcommand{\Ito}{It\^{o} }
\newcommand{\Itos}{It\^{o}'s }
\newcommand{\spt}[1]{\text{supp}\left(#1\right)}
\newcommand{\InPrd}[1]{\left\langle #1 \right\rangle}
\newcommand{\mr}{\textbf{r}}
\newcommand{\Ei}{\text{Ei}}
\newcommand{\arctanh}{\operatorname{arctanh}}
\newcommand{\ind}[1]{\mathbb{I}_{\left\{ {#1} \right\} }}
\newcommand{\Var}{\text{Var}}
\newcommand{\Cov}{\text{Cov}}
\newcommand{\Corr}{\text{Corr}}
\newcommand{\W}{\dot{W}}

\newcommand{\baseurl}[1]{\footnotesize\url{http://math.emory.edu/~lchen41/teaching/2020_Spring/#1}}

\def\bD{\mathbb{D}}
\def\bR{\mathbb{R}}
\def\bC{\mathbb{C}}
\def\e{\varepsilon}
\def\E{\mathbb{E}}
\def\P{\mathbb{P}}

\DeclareMathOperator{\esssup}{\ensuremath{ess\,sup}}
\DeclareMathOperator{\sign}{\ensuremath{sign}}

\newcommand{\steps}[1]{\vskip 0.3cm \textbf{#1}}
\newcommand{\calB}{\mathcal{B}}
\newcommand{\calC}{\mathcal{C}}
\newcommand{\calD}{\mathcal{D}}
\newcommand{\calE}{\mathcal{E}}
\newcommand{\calF}{\mathcal{F}}
\newcommand{\calG}{\mathcal{G}}
\newcommand{\calK}{\mathcal{K}}
\newcommand{\calH}{\mathcal{H}}
\newcommand{\calI}{\mathcal{I}}
\newcommand{\calL}{\mathcal{L}}
\newcommand{\calM}{\mathcal{M}}
\newcommand{\calN}{\mathcal{N}}
\newcommand{\calO}{\mathcal{O}}
\newcommand{\calT}{\mathcal{T}}
\newcommand{\calP}{\mathcal{P}}
\newcommand{\calR}{\mathcal{R}}
\newcommand{\calS}{\mathcal{S}}
\newcommand{\calV}{\mathcal{V}}
\newcommand{\bbC}{\mathbb{C}}
\newcommand{\bbN}{\mathbb{N}}
\newcommand{\bbP}{\mathbb{P}}
\newcommand{\bbZ}{\mathbb{Z}}
\newcommand{\myVec}[1]{\overrightarrow{#1}}
\newcommand{\sincos}{\begin{array}{c} \cos \\ \sin \end{array}\!\!}
\newcommand{\CvBc}[1]{\left\{\:#1\:\right\}}
\newcommand*{\one}{{{\rm 1\mkern-1.5mu}\!{\rm I}}}
\def\e{{\rm e}}
\def\cA{\mathcal{A}}
\def\cB{\mathcal{B}}
\def\cC{\mathcal{C}}
\def\cD{\mathcal{D}}
\def\cE{\mathcal{E}}
\def\cF{\mathcal{F}}
\def\cG{\mathcal{G}}
\def\cH{\mathcal{H}}
\def\cI{\mathcal{I}}
\def\cL{\mathcal{L}}
\def\cM{\mathcal{M}}
\def\cP{\mathcal{P}}
\def\cQ{\mathcal{Q}}
\def\cS{\mathcal{S}}
\def\cU{\mathcal{U}}

\newcommand{\OneFrame}[1]{
\begin{enumerate}\item[#1] \phantom{av} \\[20em]\vfill\phantom{av}\myEnd\end{enumerate}}

\newcommand{\bH}{\ensuremath{\mathrm{H}}}
\newcommand{\Ai}{\ensuremath{\mathrm{Ai}}}

\newcommand{\R}{\mathbb{R}}
\newcommand{\myEnd}{\hfill$\square$}
\newcommand{\ds}{\displaystyle}
\newcommand{\Shi}{\text{Shi}}
\newcommand{\Chi}{\text{Chi}}
\newcommand{\Erf}{\ensuremath{\mathrm{erf}}}
\newcommand{\Erfc}{\ensuremath{\mathrm{erfc}}}
\newcommand{\He}{\ensuremath{\mathrm{He}}}
\newcommand{\Res}{\ensuremath{\mathrm{Res}}}

\newcommand{\FoxH}[5]{H_{#2}^{#1}\left(#3\:\middle\vert\: \begin{subarray}{l}#4\\[0.4em] #5\end{subarray}\right)}
\newcommand{\mySeparateLine}{\begin{center}
 \makebox[\linewidth]{\rule{0.6\paperwidth}{0.4pt}}
\end{center}}
\newcommand{\myfootnoteline}{\noindent\rule{0.3\textwidth}{0.4pt}\\ \bigskip}

\theoremstyle{definition}
% \newtheorem{definition}[theorem]{Definition}
% \newtheorem{hypothesis}[theorem]{Hypothesis}
\newtheorem{assumption}[theorem]{Assumption}

\theoremstyle{plain}
% \newtheorem{theorem}{Theorem}
% \newtheorem{corollary}[theorem]{Corollary}
% \newtheorem{lemma}[theorem]{Lemma}
\newtheorem{proposition}[theorem]{Proposition}

\mode<presentation>
{
%      \usetheme{Warsaw}
%     \usetheme{JuanLesPins}
%  \usetheme{Hannover}
%  \usetheme{Montpellier}
   \useoutertheme{default}
  % or ...

  \setbeamercovered{transparent}
  % or whatever (possibly just delete it)
 \setbeamertemplate{frametitle}{
  \begin{centering}
    \color{blue}
    {\insertframetitle}
    \par
  \end{centering}
  }
}
\usefoottemplate{\hfill \insertframenumber{}}
% \inserttotalframenumber

\usepackage[english]{babel}
% or whatever

% \usepackage[latin1]{inputenc}
% or whatever

\usepackage{times}
\usepackage[T1]{fontenc}
% Or whatever. Note that the encoding and the font should match. If T1
% does not look nice, try deleting the line with the fontenc.

% \DeclareMathOperator{\Lip}{Lip}
\DeclareMathOperator{\lip}{l}
% \DeclareMathOperator{\Vip}{\overline{v}}
% \DeclareMathOperator{\vip}{\underline{v}}
% \DeclareMathOperator{\vv}{v}
% \DeclareMathOperator{\BC}{BC}
% \DeclareMathOperator{\CH}{CD}

\usepackage{pgfpages}
% \setbeameroption{show notes}
% \setbeamertemplate{note page}[plain]
% \setbeameroption{second mode text on second screen=right}
% \setbeameroption{show notes on second screen=right}
%

% Delete this, if you do not want the table of contents to pop up at
% the beginning of each subsection:
% \AtBeginSubsection[]
% {
%   \begin{frame}<beamer>{Outline}
%     \tableofcontents[currentsection,currentsubsection]
%   \end{frame}
% }

% If you wish to uncover everything in a step-wise fashion, uncomment
% the following command:
% \beamerdefaultoverlayspecification{<+->}

% % % % % % % % % % % % % % % % % % %
%  Define a block
% % % % % % % % % % % % % % % % % % %
\newenvironment<>{problock}[1]{%
  \begin{actionenv}#2%
      \def\insertblocktitle{#1}%
      \par%
      \mode<presentation>{%
        \setbeamercolor{block title}{fg=white,bg=olive!95!black}
       \setbeamercolor{block body}{fg=black,bg=olive!25!white}
       \setbeamercolor{itemize item}{fg=white!20!white}
       \setbeamertemplate{itemize item}[triangle]
     }%
      \usebeamertemplate{block begin}}
    {\par\usebeamertemplate{block end}\end{actionenv}}

\newenvironment<>{assblock}[1]{%
  \begin{actionenv}#2%
      \def\insertblocktitle{#1}%
      \par%
      \mode<presentation>{%
        \setbeamercolor{block title}{fg=white,bg=green!50!black}
       \setbeamercolor{block body}{fg=black,bg=green!10}
       \setbeamercolor{itemize item}{fg=green!80!black}
       \setbeamertemplate{itemize item}[triangle]
     }%
      \usebeamertemplate{block begin}}
    {\par\usebeamertemplate{block end}\end{actionenv}}


% \newtheorem{proofnoend}{Proof.}
% \AtBeginEnvironment{proofnoend}{%
%   \setbeamercolor{block title}{use=example text,fg=lgtblue,bg=background}
%   % \setbeamercolor{block body}{parent=normal text,use=block title example,fg=yellow}
% }

% Define some colors
\definecolor{white}{HTML}{FFFFFF}              % #FFFFFF
\definecolor{pink}{HTML}{FB73BE}               % #FB73BE
\definecolor{coral}{HTML}{FF8D71}              % #FF8D71
% \definecolor{yellow}{HTML}{9A7D0A}             % #9A7D0A
\definecolor{yellow}{HTML}{6E2C00}             % #6E2C00
% \definecolor{yellow}{HTML}{7D6608}             % #7D6608
% \definecolor{yellow}{HTML}{7E5109}             % #7E5109
% \definecolor{yellow}{HTML}{FFE066}             % #FFE066
\definecolor{teal}{HTML}{59F3CE}               % #59F3CE
\definecolor{lgtblue}{HTML}{2980B9} 	       % #1A5276
% \definecolor{lgtblue}{HTML}{65D0FA} 	       % #65D0FA
\definecolor{blue}{HTML}{4984F2}               % #4984F2
\definecolor{purple}{HTML}{A87DFF}             % #A87DFF
\definecolor{red}{HTML}{FF3d30}                % #FF3d30
% \definecolor{magenta}{HTML}{FF80FF}                % #FF3d30
% \definecolor{green}{HTML}{145A32}              % #145A32
\definecolor{green}{HTML}{1D8348}              % #145A32
% \definecolor{green}{HTML}{59F3CE}              % #59F3CE
\setbeamercolor{alerted text}{fg=red}
% \setbeamercolor{block title}{bg=background,fg=lgtblue}

\setbeamercolor{section in toc}{fg=black}
\setbeamercolor{subsection in toc}{fg=red}

% \newtheorem{myexample}{\it Example}[section]
\newcounter{myexample}[section]
\resetcounteronoverlays{myexample}
\newenvironment{myexample}[1][]{\refstepcounter{myexample}\par\medskip
\noindent \textbf{\textcolor{green}{Example~\mySecNum-\themyexample~#1}} \rmfamily}{\medskip}

\newcounter{mydefinition}[section]
\resetcounteronoverlays{mydefinition}
\newenvironment{mydefinition}[1][]{\refstepcounter{mydefinition}\par\medskip
\noindent \textbf{\textcolor{yellow}{Definition~\mySecNum-\themydefinition~#1}} \rmfamily}{\medskip}

% \NewCommandCopy{\oldref}{\ref}
% \let\oldref\ref
% \newcommand{\myref}[1]{\mySecNum-\ref{#1}}

% \newcounter{remark}[section]
% \resetcounteronoverlays{remark}
% \newenvironment{remark}[1][]{\refstepcounter{remark}\par\medskip
% \noindent \textbf{\textcolor{blue}{Remark~\mySecNum-\theremark~#1}} \rmfamily}{\medskip}

\newcounter{mythm}[section]
\resetcounteronoverlays{mythm}
\newenvironment{mythm}[1][]{\refstepcounter{mythm}\par\medskip
\noindent \textbf{\textcolor{lgtblue}{Theorem~\mySecNum-\themythm~#1}} \rmfamily}{\medskip}

\newenvironment{mycor}[1][]{\refstepcounter{mythm}\par\medskip
\noindent \textbf{\textcolor{lgtblue}{Corollary~\mySecNum-\themythm~#1}} \rmfamily}{\medskip}

% \newtheorem{solution}{\textcolor{purple}{Solution}}
\newenvironment{mysol}[1][]{\par\medskip
\noindent \textbf{\textcolor{purple}{Solution#1.~}} \rmfamily}{\medskip}

\long\def\script#1{}


% -------------------------------------------------
%  The followings are general set up for my beamer
% -------------------------------------------------
\setbeamercovered{invisible} % \includeonlyframes{current}
 %  \documentclass[compress,trans,9pt,table]{beamer}
% \documentclass[compress,9pt,usenames,dvipsnames]{beamer}
% \documentclass[9pt]{beamer}
% \usepackage[utf8]{inputenc}
\setbeamercovered{dynamic}
\usepackage{etex}
\usepackage{multirow}
\usepackage{amsmath}
\usepackage{multicol}
\usepackage{graphicx,url,psfrag}
\usepackage{tikz}
\usetikzlibrary{
  arrows,
  arrows.meta,
  calc,
  decorations.fractals,
  decorations.pathreplacing,
  graphs,
  mindmap,
  positioning,
  shapes.geometric,
  through,shapes,patterns,
  trees,
  decorations.text,
  fadings,
  }
\usepackage{smartdiagram}
\usepackage{comment}
\excludecomment{codes}
\usesmartdiagramlibrary{additions}
\usepackage[center]{subfigure}
\usepackage{enumerate}
\usepackage[makeroom]{cancel}
\newcommand{\Crossme}[1]{\!\!
\tikz [black,x=1.1em,y=1.1em,line width=.4ex]
\draw (-0.5,-0.5) -- (0,0) node {\footnotesize #1} -- (0.5,0.5) (0.5,-0.5) -- (-0.5,0.5);}%
\newcommand{\Checkme}[1]{\!\!
\tikz [x=1.1em,y=1.1em,line width=.4ex]
\draw [black] (0,0.7) -- (0.3,0) --(0.9,1.0) (0.5,0.5) node {\footnotesize #1};}
\beamerdefaultoverlayspecification{<+-| alert@+>} %(this will show line by line)
\beamerdefaultoverlayspecification{<+->} %(this will show line by
\usepackage{arydshln}
% Smiley face\Smiley{} \Frowny{}
\usepackage{marvosym}

\usepackage{pifont}
\newcommand{\xmark}{\ding{55}}%

% pause equations
\makeatletter
\renewrobustcmd{\beamer@@pause}[1][]{%
  \unless\ifmeasuring@%
  \ifblank{#1}%
    {\stepcounter{beamerpauses}}%
    {\setcounter{beamerpauses}{#1}}%
  \onslide<\value{beamerpauses}->\relax%
  \fi%
}
\makeatother

% -------------------------------------------------
%  Set directory for figs
% -------------------------------------------------
\usepackage{grffile}
\graphicspath{{../2016-11-29-McGill/figs/}}

% -------------------------------------------------
%  Define colors
% -------------------------------------------------
% Darkmode
% \usepackage{dbt}
% \def\refcolor{green(html/cssgreen)}
\newcommand{\myref}[1]{\small {\em #1}}
% \def\excolor{brown}
\usepackage{color}
\usepackage{xcolor}
\definecolor{airforceblue}{rgb}{0.36, 0.54, 0.66}             % #5D8AA8
\definecolor{aliceblue}{rgb}{0.94, 0.97, 1.0}                 % #F0F8FF
\definecolor{alizarin}{rgb}{0.82, 0.1, 0.26}                  % #E32636
\definecolor{almond}{rgb}{0.94, 0.87, 0.8}                    % #EFDECD
\definecolor{amaranth}{rgb}{0.9, 0.17, 0.31}                  % #E52B50
\definecolor{amber}{rgb}{1.0, 0.75, 0.0}                      % #FFBF00
\definecolor{amber(sae/ece)}{rgb}{1.0, 0.49, 0.0}             % #FF7E00
\definecolor{americanrose}{rgb}{1.0, 0.01, 0.24}              % #FF033E
\definecolor{amethyst}{rgb}{0.6, 0.4, 0.8}                    % #9966CC
\definecolor{anti-flashwhite}{rgb}{0.95, 0.95, 0.96}          % #F2F3F4
\definecolor{antiquebrass}{rgb}{0.8, 0.58, 0.46}              % #CD9575
\definecolor{antiquefuchsia}{rgb}{0.57, 0.36, 0.51}           % #915C83
\definecolor{antiquewhite}{rgb}{0.98, 0.92, 0.84}             % #FAEBD7
\definecolor{ao}{rgb}{0.0, 0.0, 1.0}                          % #0000FF
\definecolor{ao(english)}{rgb}{0.0, 0.5, 0.0}                 % #008000
\definecolor{applegreen}{rgb}{0.55, 0.71, 0.0}                % #8DB600
\definecolor{apricot}{rgb}{0.98, 0.81, 0.69}                  % #FBCEB1
\definecolor{aqua}{rgb}{0.0, 1.0, 1.0}                        % #00FFFF
\definecolor{aquamarine}{rgb}{0.5, 1.0, 0.83}                 % #7FFFD0
\definecolor{armygreen}{rgb}{0.29, 0.33, 0.13}                % #4B5320
\definecolor{arsenic}{rgb}{0.23, 0.27, 0.29}                  % #3B444B
\definecolor{arylideyellow}{rgb}{0.91, 0.84, 0.42}            % #E9D66B
\definecolor{ashgrey}{rgb}{0.7, 0.75, 0.71}                   % #B2BEB5
\definecolor{asparagus}{rgb}{0.53, 0.66, 0.42}                % #87A96B
\definecolor{atomictangerine}{rgb}{1.0, 0.6, 0.4}             % #FF9966
\definecolor{auburn}{rgb}{0.43, 0.21, 0.1}                    % #6D351A
\definecolor{aureolin}{rgb}{0.99, 0.93, 0.0}                  % #FDEE00
\definecolor{aurometalsaurus}{rgb}{0.43, 0.5, 0.5}            % #6E7F80
\definecolor{awesome}{rgb}{1.0, 0.13, 0.32}                   % #FF2052
\definecolor{azure(colorwheel)}{rgb}{0.0, 0.5, 1.0}           % #007FFF
\definecolor{azure(web)(azuremist)}{rgb}{0.94, 1.0, 1.0}      % #F0FFFF
\definecolor{babyblue}{rgb}{0.54, 0.81, 0.94}                 % #89CFF0
\definecolor{babyblueeyes}{rgb}{0.63, 0.79, 0.95}             % #A1CAF1
\definecolor{babypink}{rgb}{0.96, 0.76, 0.76}                 % #F4C2C2
\definecolor{ballblue}{rgb}{0.13, 0.67, 0.8}                  % #21ABCD
\definecolor{bananamania}{rgb}{0.98, 0.91, 0.71}              % #FAE7B5
\definecolor{bananayellow}{rgb}{1.0, 0.88, 0.21}              % #FFE135
\definecolor{battleshipgrey}{rgb}{0.52, 0.52, 0.51}           % #848482
\definecolor{bazaar}{rgb}{0.6, 0.47, 0.48}                    % #98777B
\definecolor{beaublue}{rgb}{0.74, 0.83, 0.9}                  % #BCD4E6
\definecolor{beaver}{rgb}{0.62, 0.51, 0.44}                   % #9F8170
\definecolor{beige}{rgb}{0.96, 0.96, 0.86}                    % #F5F5DC
\definecolor{bisque}{rgb}{1.0, 0.89, 0.77}                    % #FFE4C4
\definecolor{bistre}{rgb}{0.24, 0.17, 0.12}                   % #3D2B1F
\definecolor{bittersweet}{rgb}{1.0, 0.44, 0.37}               % #FE6F5E
\definecolor{black}{rgb}{0.0, 0.0, 0.0}                       % #000000
\definecolor{blanchedalmond}{rgb}{1.0, 0.92, 0.8}             % #FFEBCD
\definecolor{bleudefrance}{rgb}{0.19, 0.55, 0.91}             % #318CE7
\definecolor{blizzardblue}{rgb}{0.67, 0.9, 0.93}              % #ACE5EE
\definecolor{blond}{rgb}{0.98, 0.94, 0.75}                    % #FAF0BE
\definecolor{blue}{rgb}{0.0, 0.0, 1.0}                        % #0000FF
\definecolor{blue(munsell)}{rgb}{0.0, 0.5, 0.69}              % #0093AF
\definecolor{blue(ncs)}{rgb}{0.0, 0.53, 0.74}                 % #0087BD
\definecolor{blue(pigment)}{rgb}{0.2, 0.2, 0.6}               % #333399
\definecolor{blue(ryb)}{rgb}{0.01, 0.28, 1.0}                 % #0247FE
\definecolor{bluebell}{rgb}{0.64, 0.64, 0.82}                 % #A2A2D0
\definecolor{bluegray}{rgb}{0.4, 0.6, 0.8}                    % #6699CC
\definecolor{blue-green}{rgb}{0.0, 0.87, 0.87}                % #00DDDD
\definecolor{blue-violet}{rgb}{0.54, 0.17, 0.89}              % #8A2BE2
\definecolor{blush}{rgb}{0.87, 0.36, 0.51}                    % #DE5D83
\definecolor{bole}{rgb}{0.47, 0.27, 0.23}                     % #79443B
\definecolor{bondiblue}{rgb}{0.0, 0.58, 0.71}                 % #0095B6
\definecolor{bostonuniversityred}{rgb}{0.8, 0.0, 0.0}         % #CC0000
\definecolor{brandeisblue}{rgb}{0.0, 0.44, 1.0}               % #0070FF
\definecolor{brass}{rgb}{0.71, 0.65, 0.26}                    % #B5A642
\definecolor{brickred}{rgb}{0.8, 0.25, 0.33}                  % #CB4154
\definecolor{brightcerulean}{rgb}{0.11, 0.67, 0.84}           % #1DACD6
\definecolor{brightgreen}{rgb}{0.4, 1.0, 0.0}                 % #66FF00
\definecolor{brightlavender}{rgb}{0.75, 0.58, 0.89}           % #BF94E4
\definecolor{brightmaroon}{rgb}{0.76, 0.13, 0.28}             % #C32148
\definecolor{brightpink}{rgb}{1.0, 0.0, 0.5}                  % #FF007F
\definecolor{brightturquoise}{rgb}{0.03, 0.91, 0.87}          % #08E8DE
\definecolor{brightube}{rgb}{0.82, 0.62, 0.91}                % #D19FE8
\definecolor{brilliantlavender}{rgb}{0.96, 0.73, 1.0}         % #F4BBFF
\definecolor{brilliantrose}{rgb}{1.0, 0.33, 0.64}             % #FF55A3
\definecolor{brinkpink}{rgb}{0.98, 0.38, 0.5}                 % #FB607F
\definecolor{britishracinggreen}{rgb}{0.0, 0.26, 0.15}        % #004225
\definecolor{bronze}{rgb}{0.8, 0.5, 0.2}                      % #CD7F32
\definecolor{brown(traditional)}{rgb}{0.59, 0.29, 0.0}        % #964B00
\definecolor{brown(web)}{rgb}{0.65, 0.16, 0.16}               % #A52A2A
\definecolor{bubblegum}{rgb}{0.99, 0.76, 0.8}                 % #FFC1CC
\definecolor{bubbles}{rgb}{0.91, 1.0, 1.0}                    % #E7FEFF
\definecolor{buff}{rgb}{0.94, 0.86, 0.51}                     % #F0DC82
\definecolor{bulgarianrose}{rgb}{0.28, 0.02, 0.03}            % #480607
\definecolor{burgundy}{rgb}{0.5, 0.0, 0.13}                   % #800020
\definecolor{burlywood}{rgb}{0.87, 0.72, 0.53}                % #DEB887
\definecolor{burntorange}{rgb}{0.8, 0.33, 0.0}                % #CC5500
\definecolor{burntsienna}{rgb}{0.91, 0.45, 0.32}              % #E97451
\definecolor{burntumber}{rgb}{0.54, 0.2, 0.14}                % #8A3324
\definecolor{byzantine}{rgb}{0.74, 0.2, 0.64}                 % #BD33A4
\definecolor{byzantium}{rgb}{0.44, 0.16, 0.39}                % #702963
\definecolor{cadet}{rgb}{0.33, 0.41, 0.47}                    % #536872
\definecolor{cadetblue}{rgb}{0.37, 0.62, 0.63}                % #5F9EA0
\definecolor{cadetgrey}{rgb}{0.57, 0.64, 0.69}                % #91A3B0
\definecolor{cadmiumgreen}{rgb}{0.0, 0.42, 0.24}              % #006B3C
\definecolor{cadmiumorange}{rgb}{0.93, 0.53, 0.18}            % #ED872D
\definecolor{cadmiumred}{rgb}{0.89, 0.0, 0.13}                % #E30022
\definecolor{cadmiumyellow}{rgb}{1.0, 0.96, 0.0}              % #FFF600
\definecolor{calpolypomonagreen}{rgb}{0.12, 0.3, 0.17}        % #1E4D2B
\definecolor{cambridgeblue}{rgb}{0.64, 0.76, 0.68}            % #A3C1AD
\definecolor{camel}{rgb}{0.76, 0.6, 0.42}                     % #C19A6B
\definecolor{camouflagegreen}{rgb}{0.47, 0.53, 0.42}          % #78866B
\definecolor{canaryyellow}{rgb}{1.0, 0.94, 0.0}               % #FFEF00
\definecolor{candyapplered}{rgb}{1.0, 0.03, 0.0}              % #FF0800
\definecolor{candypink}{rgb}{0.89, 0.44, 0.48}                % #E4717A
\definecolor{capri}{rgb}{0.0, 0.75, 1.0}                      % #00BFFF
\definecolor{caputmortuum}{rgb}{0.35, 0.15, 0.13}             % #592720
\definecolor{cardinal}{rgb}{0.77, 0.12, 0.23}                 % #C41E3A
\definecolor{caribbeangreen}{rgb}{0.0, 0.8, 0.6}              % #00CC99
\definecolor{carmine}{rgb}{0.59, 0.0, 0.09}                   % #960018
\definecolor{carminepink}{rgb}{0.92, 0.3, 0.26}               % #EB4C42
\definecolor{carminered}{rgb}{1.0, 0.0, 0.22}                 % #FF0038
\definecolor{carnationpink}{rgb}{1.0, 0.65, 0.79}             % #FFA6C9
\definecolor{carnelian}{rgb}{0.7, 0.11, 0.11}                 % #B31B1B
\definecolor{carolinablue}{rgb}{0.6, 0.73, 0.89}              % #99BADD
\definecolor{carrotorange}{rgb}{0.93, 0.57, 0.13}             % #ED9121
\definecolor{ceil}{rgb}{0.57, 0.63, 0.81}                     % #92A1CF
\definecolor{celadon}{rgb}{0.67, 0.88, 0.69}                  % #ACE1AF
\definecolor{celestialblue}{rgb}{0.29, 0.59, 0.82}            % #4997D0
\definecolor{cerise}{rgb}{0.87, 0.19, 0.39}                   % #DE3163
\definecolor{cerisepink}{rgb}{0.93, 0.23, 0.51}               % #EC3B83
\definecolor{cerulean}{rgb}{0.0, 0.48, 0.65}                  % #007BA7
\definecolor{ceruleanblue}{rgb}{0.16, 0.32, 0.75}             % #2A52BE
\definecolor{chamoisee}{rgb}{0.63, 0.47, 0.35}                % #A0785A
\definecolor{champagne}{rgb}{0.97, 0.91, 0.81}                % #F7E7CE
\definecolor{charcoal}{rgb}{0.21, 0.27, 0.31}                 % #36454F
\definecolor{chartreuse(traditional)}{rgb}{0.87, 1.0, 0.0}    % #DFFF00
\definecolor{chartreuse(web)}{rgb}{0.5, 1.0, 0.0}             % #7FFF00
\definecolor{cherryblossompink}{rgb}{1.0, 0.72, 0.77}         % #FFB7C5
\definecolor{chestnut}{rgb}{0.8, 0.36, 0.36}                  % #CD5C5C
\definecolor{chocolate(traditional)}{rgb}{0.48, 0.25, 0.0}    % #7B3F00
\definecolor{chocolate(web)}{rgb}{0.82, 0.41, 0.12}           % #D2691E
\definecolor{chromeyellow}{rgb}{1.0, 0.65, 0.0}               % #FFA700
\definecolor{cinereous}{rgb}{0.6, 0.51, 0.48}                 % #98817B
\definecolor{cinnabar}{rgb}{0.89, 0.26, 0.2}                  % #E34234
\definecolor{cinnamon}{rgb}{0.82, 0.41, 0.12}                 % #D2691E
\definecolor{citrine}{rgb}{0.89, 0.82, 0.04}                  % #E4D00A
\definecolor{classicrose}{rgb}{0.98, 0.8, 0.91}               % #FBCCE7
\definecolor{cobalt}{rgb}{0.0, 0.28, 0.67}                    % #0047AB
\definecolor{cocoabrown}{rgb}{0.82, 0.41, 0.12}               % #D2691E
\definecolor{columbiablue}{rgb}{0.61, 0.87, 1.0}              % #9BDDFF
\definecolor{coolblack}{rgb}{0.0, 0.18, 0.39}                 % #002E63
\definecolor{coolgrey}{rgb}{0.55, 0.57, 0.67}                 % #8C92AC
\definecolor{copper}{rgb}{0.72, 0.45, 0.2}                    % #B87333
\definecolor{copperrose}{rgb}{0.6, 0.4, 0.4}                  % #996666
\definecolor{coquelicot}{rgb}{1.0, 0.22, 0.0}                 % #FF3800
\definecolor{coral}{rgb}{1.0, 0.5, 0.31}                      % #FF7F50
\definecolor{coralpink}{rgb}{0.97, 0.51, 0.47}                % #F88379
\definecolor{coralred}{rgb}{1.0, 0.25, 0.25}                  % #FF4040
\definecolor{cordovan}{rgb}{0.54, 0.25, 0.27}                 % #893F45
\definecolor{corn}{rgb}{0.98, 0.93, 0.36}                     % #FBEC5D
\definecolor{cornellred}{rgb}{0.7, 0.11, 0.11}                % #B31B1B
\definecolor{cornflowerblue}{rgb}{0.39, 0.58, 0.93}           % #6495ED
\definecolor{cornsilk}{rgb}{1.0, 0.97, 0.86}                  % #FFF8DC
\definecolor{cosmiclatte}{rgb}{1.0, 0.97, 0.91}               % #FFF8E7
\definecolor{cottoncandy}{rgb}{1.0, 0.74, 0.85}               % #FFBCD9
\definecolor{cream}{rgb}{1.0, 0.99, 0.82}                     % #FFFDD0
\definecolor{crimson}{rgb}{0.86, 0.08, 0.24}                  % #DC143C
\definecolor{crimsonglory}{rgb}{0.75, 0.0, 0.2}               % #BE0032
\definecolor{cyan}{rgb}{0.0, 1.0, 1.0}                        % #00FFFF
\definecolor{cyan(process)}{rgb}{0.0, 0.72, 0.92}             % #00B7EB
\definecolor{daffodil}{rgb}{1.0, 1.0, 0.19}                   % #FFFF31
\definecolor{dandelion}{rgb}{0.94, 0.88, 0.19}                % #F0E130
\definecolor{darkblue}{rgb}{0.0, 0.0, 0.55}                   % #00008B
\definecolor{darkbrown}{rgb}{0.4, 0.26, 0.13}                 % #654321
\definecolor{darkbyzantium}{rgb}{0.36, 0.22, 0.33}            % #5D3954
\definecolor{darkcandyapplered}{rgb}{0.64, 0.0, 0.0}          % #A40000
\definecolor{darkcerulean}{rgb}{0.03, 0.27, 0.49}             % #08457E
\definecolor{darkchampagne}{rgb}{0.76, 0.7, 0.5}              % #C2B280
\definecolor{darkchestnut}{rgb}{0.6, 0.41, 0.38}              % #986960
\definecolor{darkcoral}{rgb}{0.8, 0.36, 0.27}                 % #CD5B45
\definecolor{darkcyan}{rgb}{0.0, 0.55, 0.55}                  % #008B8B
\definecolor{darkelectricblue}{rgb}{0.33, 0.41, 0.47}         % #536878
\definecolor{darkgoldenrod}{rgb}{0.72, 0.53, 0.04}            % #B8860B
\definecolor{darkgray}{rgb}{0.66, 0.66, 0.66}                 % #A9A9A9
\definecolor{darkgreen}{rgb}{0.0, 0.2, 0.13}                  % #013220
\definecolor{darkjunglegreen}{rgb}{0.1, 0.14, 0.13}           % #1A2421
\definecolor{darkkhaki}{rgb}{0.74, 0.72, 0.42}                % #BDB76B
\definecolor{darklava}{rgb}{0.28, 0.24, 0.2}                  % #483C32
\definecolor{darklavender}{rgb}{0.45, 0.31, 0.59}             % #734F96
\definecolor{darkmagenta}{rgb}{0.55, 0.0, 0.55}               % #8B008B
\definecolor{darkmidnightblue}{rgb}{0.0, 0.2, 0.4}            % #003366
\definecolor{darkolivegreen}{rgb}{0.33, 0.42, 0.18}           % #556B2F
\definecolor{darkorange}{rgb}{1.0, 0.55, 0.0}                 % #FF8C00
\definecolor{darkorchid}{rgb}{0.6, 0.2, 0.8}                  % #9932CC
\definecolor{darkpastelblue}{rgb}{0.47, 0.62, 0.8}            % #779ECB
\definecolor{darkpastelgreen}{rgb}{0.01, 0.75, 0.24}          % #03C03C
\definecolor{darkpastelpurple}{rgb}{0.59, 0.44, 0.84}         % #966FD6
\definecolor{darkpastelred}{rgb}{0.76, 0.23, 0.13}            % #C23B22
\definecolor{darkpink}{rgb}{0.91, 0.33, 0.5}                  % #E75480
\definecolor{darkpowderblue}{rgb}{0.0, 0.2, 0.6}              % #003399
\definecolor{darkraspberry}{rgb}{0.53, 0.15, 0.34}            % #872657
\definecolor{darkred}{rgb}{0.55, 0.0, 0.0}                    % #8B0000
\definecolor{darksalmon}{rgb}{0.91, 0.59, 0.48}               % #E9967A
\definecolor{darkscarlet}{rgb}{0.34, 0.01, 0.1}               % #560319
\definecolor{darkseagreen}{rgb}{0.56, 0.74, 0.56}             % #8FBC8F
\definecolor{darksienna}{rgb}{0.24, 0.08, 0.08}               % #3C1414
\definecolor{darkslateblue}{rgb}{0.28, 0.24, 0.55}            % #483D8B
\definecolor{darkslategray}{rgb}{0.18, 0.31, 0.31}            % #2F4F4F
\definecolor{darkspringgreen}{rgb}{0.09, 0.45, 0.27}          % #177245
\definecolor{darktan}{rgb}{0.57, 0.51, 0.32}                  % #918151
\definecolor{darktangerine}{rgb}{1.0, 0.66, 0.07}             % #FFA812
\definecolor{darktaupe}{rgb}{0.28, 0.24, 0.2}                 % #483C32
\definecolor{darkterracotta}{rgb}{0.8, 0.31, 0.36}            % #CC4E5C
\definecolor{darkturquoise}{rgb}{0.0, 0.81, 0.82}             % #00CED1
\definecolor{darkviolet}{rgb}{0.58, 0.0, 0.83}                % #9400D3
\definecolor{dartmouthgreen}{rgb}{0.05, 0.5, 0.06}            % #00693E
\definecolor{davy\'sgrey}{rgb}{0.33, 0.33, 0.33}              % #555555
\definecolor{debianred}{rgb}{0.84, 0.04, 0.33}                % #D70A53
\definecolor{deepcarmine}{rgb}{0.66, 0.13, 0.24}              % #A9203E
\definecolor{deepcarminepink}{rgb}{0.94, 0.19, 0.22}          % #EF3038
\definecolor{deepcarrotorange}{rgb}{0.91, 0.41, 0.17}         % #E9692C
\definecolor{deepcerise}{rgb}{0.85, 0.2, 0.53}                % #DA3287
\definecolor{deepchampagne}{rgb}{0.98, 0.84, 0.65}            % #FAD6A5
\definecolor{deepchestnut}{rgb}{0.73, 0.31, 0.28}             % #B94E48
\definecolor{deepfuchsia}{rgb}{0.76, 0.33, 0.76}              % #C154C1
\definecolor{deepjunglegreen}{rgb}{0.0, 0.29, 0.29}           % #004B49
\definecolor{deeplilac}{rgb}{0.6, 0.33, 0.73}                 % #9955BB
\definecolor{deepmagenta}{rgb}{0.8, 0.0, 0.8}                 % #CC00CC
\definecolor{deeppeach}{rgb}{1.0, 0.8, 0.64}                  % #FFCBA4
\definecolor{deeppink}{rgb}{1.0, 0.08, 0.58}                  % #FF1493
\definecolor{deepsaffron}{rgb}{1.0, 0.6, 0.2}                 % #FF9933
\definecolor{deepskyblue}{rgb}{0.0, 0.75, 1.0}                % #00BFFF
\definecolor{denim}{rgb}{0.08, 0.38, 0.74}                    % #1560BD
\definecolor{desert}{rgb}{0.76, 0.6, 0.42}                    % #C19A6B
\definecolor{desertsand}{rgb}{0.93, 0.79, 0.69}               % #EDC9AF
\definecolor{dimgray}{rgb}{0.41, 0.41, 0.41}                  % #696969
\definecolor{dodgerblue}{rgb}{0.12, 0.56, 1.0}                % #1E90FF
\definecolor{dogwoodrose}{rgb}{0.84, 0.09, 0.41}              % #D71868
\definecolor{dollarbill}{rgb}{0.52, 0.73, 0.4}                % #85BB65
\definecolor{drab}{rgb}{0.59, 0.44, 0.09}                     % #967117
\definecolor{dukeblue}{rgb}{0.0, 0.0, 0.61}                   % #00009C
\definecolor{earthyellow}{rgb}{0.88, 0.66, 0.37}              % #E1A95F
\definecolor{ecru}{rgb}{0.76, 0.7, 0.5}                       % #C2B280
\definecolor{eggplant}{rgb}{0.38, 0.25, 0.32}                 % #614051
\definecolor{eggshell}{rgb}{0.94, 0.92, 0.84}                 % #F0EAD6
\definecolor{egyptianblue}{rgb}{0.06, 0.2, 0.65}              % #1034A6
\definecolor{electricblue}{rgb}{0.49, 0.98, 1.0}              % #7DF9FF
\definecolor{electriccrimson}{rgb}{1.0, 0.0, 0.25}            % #FF003F
\definecolor{electriccyan}{rgb}{0.0, 1.0, 1.0}                % #00FFFF
\definecolor{electricgreen}{rgb}{0.0, 1.0, 0.0}               % #00FF00
\definecolor{electricindigo}{rgb}{0.44, 0.0, 1.0}             % #6F00FF
\definecolor{electriclavender}{rgb}{0.96, 0.73, 1.0}          % #F4BBFF
\definecolor{electriclime}{rgb}{0.8, 1.0, 0.0}                % #CCFF00
\definecolor{electricpurple}{rgb}{0.75, 0.0, 1.0}             % #BF00FF
\definecolor{electricultramarine}{rgb}{0.25, 0.0, 1.0}        % #3F00FF
\definecolor{electricviolet}{rgb}{0.56, 0.0, 1.0}             % #8F00FF
\definecolor{electricyellow}{rgb}{1.0, 1.0, 0.0}              % #FFFF00
\definecolor{emerald}{rgb}{0.31, 0.78, 0.47}                  % #50C878
\definecolor{etonblue}{rgb}{0.59, 0.78, 0.64}                 % #96C8A2
\definecolor{fallow}{rgb}{0.76, 0.6, 0.42}                    % #C19A6B
\definecolor{falured}{rgb}{0.5, 0.09, 0.09}                   % #801818
\definecolor{fandango}{rgb}{0.71, 0.2, 0.54}                  % #B53389
\definecolor{fashionfuchsia}{rgb}{0.96, 0.0, 0.63}            % #F400A1
\definecolor{fawn}{rgb}{0.9, 0.67, 0.44}                      % #E5AA70
\definecolor{feldgrau}{rgb}{0.3, 0.36, 0.33}                  % #4D5D53
\definecolor{ferngreen}{rgb}{0.31, 0.47, 0.26}                % #4F7942
\definecolor{ferrarired}{rgb}{1.0, 0.11, 0.0}                 % #FF2800
\definecolor{fielddrab}{rgb}{0.42, 0.33, 0.12}                % #6C541E
\definecolor{firebrick}{rgb}{0.7, 0.13, 0.13}                 % #B22222
\definecolor{fireenginered}{rgb}{0.81, 0.09, 0.13}            % #CE2029
\definecolor{flame}{rgb}{0.89, 0.35, 0.13}                    % #E25822
\definecolor{flamingopink}{rgb}{0.99, 0.56, 0.67}             % #FC8EAC
\definecolor{flavescent}{rgb}{0.97, 0.91, 0.56}               % #F7E98E
\definecolor{flax}{rgb}{0.93, 0.86, 0.51}                     % #EEDC82
\definecolor{floralwhite}{rgb}{1.0, 0.98, 0.94}               % #FFFAF0
\definecolor{fluorescentorange}{rgb}{1.0, 0.75, 0.0}          % #FFBF00
\definecolor{fluorescentpink}{rgb}{1.0, 0.08, 0.58}           % #FF1493
\definecolor{fluorescentyellow}{rgb}{0.8, 1.0, 0.0}           % #CCFF00
\definecolor{folly}{rgb}{1.0, 0.0, 0.31}                      % #FF004F
\definecolor{forestgreen(traditional)}{rgb}{0.0, 0.27, 0.13}  % #014421
\definecolor{forestgreen(web)}{rgb}{0.13, 0.55, 0.13}         % #228B22
\definecolor{frenchbeige}{rgb}{0.65, 0.48, 0.36}              % #A67B5B
\definecolor{frenchblue}{rgb}{0.0, 0.45, 0.73}                % #0072BB
\definecolor{frenchlilac}{rgb}{0.53, 0.38, 0.56}              % #86608E
\definecolor{frenchrose}{rgb}{0.96, 0.29, 0.54}               % #F64A8A
\definecolor{fuchsia}{rgb}{1.0, 0.0, 1.0}                     % #FF00FF
\definecolor{fuchsiapink}{rgb}{1.0, 0.47, 1.0}                % #FF77FF
\definecolor{fulvous}{rgb}{0.86, 0.52, 0.0}                   % #E48400
\definecolor{fuzzywuzzy}{rgb}{0.8, 0.4, 0.4}                  % #CC6666
\definecolor{gainsboro}{rgb}{0.86, 0.86, 0.86}                % #DCDCDC
\definecolor{gamboge}{rgb}{0.89, 0.61, 0.06}                  % #E49B0F
\definecolor{ghostwhite}{rgb}{0.97, 0.97, 1.0}                % #F8F8FF
\definecolor{ginger}{rgb}{0.69, 0.4, 0.0}                     % #B06500
\definecolor{glaucous}{rgb}{0.38, 0.51, 0.71}                 % #6082B6
\definecolor{gold(metallic)}{rgb}{0.83, 0.69, 0.22}           % #D4AF37
\definecolor{gold(web)(golden)}{rgb}{1.0, 0.84, 0.0}          % #FFD700
\definecolor{goldenbrown}{rgb}{0.6, 0.4, 0.08}                % #996515
\definecolor{goldenpoppy}{rgb}{0.99, 0.76, 0.0}               % #FCC200
\definecolor{goldenyellow}{rgb}{1.0, 0.87, 0.0}               % #FFDF00
\definecolor{goldenrod}{rgb}{0.85, 0.65, 0.13}                % #DAA520
\definecolor{grannysmithapple}{rgb}{0.66, 0.89, 0.63}         % #A8E4A0
\definecolor{gray}{rgb}{0.5, 0.5, 0.5}                        % #808080
\definecolor{gray(html/cssgray)}{rgb}{0.5, 0.5, 0.5}          % #7F7F7F
\definecolor{gray(x11gray)}{rgb}{0.75, 0.75, 0.75}            % #BEBEBE
\definecolor{gray-asparagus}{rgb}{0.27, 0.35, 0.27}           % #465945
\definecolor{green(colorwheel)(x11green)}{rgb}{0.0, 1.0, 0.0} % #00FF00
\definecolor{green(html/cssgreen)}{rgb}{0.0, 0.5, 0.0}        % #008000
\definecolor{green(munsell)}{rgb}{0.0, 0.66, 0.47}            % #00A877
\definecolor{green(ncs)}{rgb}{0.0, 0.62, 0.42}                % #009F6B
\definecolor{green(pigment)}{rgb}{0.0, 0.65, 0.31}            % #00A550
\definecolor{green(ryb)}{rgb}{0.4, 0.69, 0.2}                 % #66B032
\definecolor{green-yellow}{rgb}{0.68, 1.0, 0.18}              % #ADFF2F
\definecolor{grullo}{rgb}{0.66, 0.6, 0.53}                    % #A99A86
\definecolor{guppiegreen}{rgb}{0.0, 1.0, 0.5}                 % #00FF7F
\definecolor{halayaube}{rgb}{0.4, 0.22, 0.33}                 % #663854
\definecolor{hanblue}{rgb}{0.27, 0.42, 0.81}                  % #446CCF
\definecolor{hanpurple}{rgb}{0.32, 0.09, 0.98}                % #5218FA
\definecolor{hansayellow}{rgb}{0.91, 0.84, 0.42}              % #E9D66B
\definecolor{harlequin}{rgb}{0.25, 1.0, 0.0}                  % #3FFF00
\definecolor{harvardcrimson}{rgb}{0.79, 0.0, 0.09}            % #C90016
\definecolor{harvestgold}{rgb}{0.85, 0.57, 0.0}               % #DA9100
\definecolor{heartgold}{rgb}{0.5, 0.5, 0.0}                   % #808000
\definecolor{heliotrope}{rgb}{0.87, 0.45, 1.0}                % #DF73FF
\definecolor{hollywoodcerise}{rgb}{0.96, 0.0, 0.63}           % #F400A1
\definecolor{honeydew}{rgb}{0.94, 1.0, 0.94}                  % #F0FFF0
\definecolor{hooker\'sgreen}{rgb}{0.0, 0.44, 0.0}             % #007000
\definecolor{hotmagenta}{rgb}{1.0, 0.11, 0.81}                % #FF1DCE
\definecolor{hotpink}{rgb}{1.0, 0.41, 0.71}                   % #FF69B4
\definecolor{huntergreen}{rgb}{0.21, 0.37, 0.23}              % #355E3B
\definecolor{iceberg}{rgb}{0.44, 0.65, 0.82}                  % #71A6D2
\definecolor{icterine}{rgb}{0.99, 0.97, 0.37}                 % #FCF75E
\definecolor{inchworm}{rgb}{0.7, 0.93, 0.36}                  % #B2EC5D
\definecolor{indiagreen}{rgb}{0.07, 0.53, 0.03}               % #138808
\definecolor{indianred}{rgb}{0.8, 0.36, 0.36}                 % #CD5C5C
\definecolor{indianyellow}{rgb}{0.89, 0.66, 0.34}             % #E3A857
\definecolor{indigo(dye)}{rgb}{0.0, 0.25, 0.42}               % #00416A
\definecolor{indigo(web)}{rgb}{0.29, 0.0, 0.51}               % #4B0082
\definecolor{internationalkleinblue}{rgb}{0.0, 0.18, 0.65}    % #002FA7
\definecolor{internationalorange}{rgb}{1.0, 0.31, 0.0}        % #FF4F00
\definecolor{iris}{rgb}{0.35, 0.31, 0.81}                     % #5A4FCF
\definecolor{isabelline}{rgb}{0.96, 0.94, 0.93}               % #F4F0EC
\definecolor{islamicgreen}{rgb}{0.0, 0.56, 0.0}               % #009000
\definecolor{ivory}{rgb}{1.0, 1.0, 0.94}                      % #FFFFF0
\definecolor{jade}{rgb}{0.0, 0.66, 0.42}                      % #00A86B
\definecolor{jasper}{rgb}{0.84, 0.23, 0.24}                   % #D73B3E
\definecolor{jazzberryjam}{rgb}{0.65, 0.04, 0.37}             % #A50B5E
\definecolor{jonquil}{rgb}{0.98, 0.85, 0.37}                  % #FADA5E
\definecolor{junebud}{rgb}{0.74, 0.85, 0.34}                  % #BDDA57
\definecolor{junglegreen}{rgb}{0.16, 0.67, 0.53}              % #29AB87
\definecolor{kellygreen}{rgb}{0.3, 0.73, 0.09}                % #4CBB17
\definecolor{khaki(html/css)(khaki)}{rgb}{0.76, 0.69, 0.57}   % #C3B091
\definecolor{khaki(x11)(lightkhaki)}{rgb}{0.94, 0.9, 0.55}    % #F0E68C
\definecolor{lasallegreen}{rgb}{0.03, 0.47, 0.19}             % #087830
\definecolor{languidlavender}{rgb}{0.84, 0.79, 0.87}          % #D6CADD
\definecolor{lapislazuli}{rgb}{0.15, 0.38, 0.61}              % #26619C
\definecolor{laserlemon}{rgb}{1.0, 1.0, 0.13}                 % #FEFE22
\definecolor{lava}{rgb}{0.81, 0.06, 0.13}                     % #CF1020
\definecolor{lavender(floral)}{rgb}{0.71, 0.49, 0.86}         % #B57EDC
\definecolor{lavender(web)}{rgb}{0.9, 0.9, 0.98}              % #E6E6FA
\definecolor{lavenderblue}{rgb}{0.8, 0.8, 1.0}                % #CCCCFF
\definecolor{lavenderblush}{rgb}{1.0, 0.94, 0.96}             % #FFF0F5
\definecolor{lavendergray}{rgb}{0.77, 0.76, 0.82}             % #C4C3D0
\definecolor{lavenderindigo}{rgb}{0.58, 0.34, 0.92}           % #9457EB
\definecolor{lavendermagenta}{rgb}{0.93, 0.51, 0.93}          % #EE82EE
\definecolor{lavendermist}{rgb}{0.9, 0.9, 0.98}               % #E6E6FA
\definecolor{lavenderpink}{rgb}{0.98, 0.68, 0.82}             % #FBAED2
\definecolor{lavenderpurple}{rgb}{0.59, 0.48, 0.71}           % #967BB6
\definecolor{lavenderrose}{rgb}{0.98, 0.63, 0.89}             % #FBA0E3
\definecolor{lawngreen}{rgb}{0.49, 0.99, 0.0}                 % #7CFC00
\definecolor{lemon}{rgb}{1.0, 0.97, 0.0}                      % #FFF700
\definecolor{lemonchiffon}{rgb}{1.0, 0.98, 0.8}               % #FFFACD
\definecolor{lightapricot}{rgb}{0.99, 0.84, 0.69}             % #FDD5B1
\definecolor{lightblue}{rgb}{0.68, 0.85, 0.9}                 % #ADD8E6
\definecolor{lightbrown}{rgb}{0.71, 0.4, 0.11}                % #B5651D
\definecolor{lightcarminepink}{rgb}{0.9, 0.4, 0.38}           % #E66771
\definecolor{lightcoral}{rgb}{0.94, 0.5, 0.5}                 % #F08080
\definecolor{lightcornflowerblue}{rgb}{0.6, 0.81, 0.93}       % #93CCEA
\definecolor{lightcyan}{rgb}{0.88, 1.0, 1.0}                  % #E0FFFF
\definecolor{lightfuchsiapink}{rgb}{0.98, 0.52, 0.9}          % #F984EF
\definecolor{lightgoldenrodyellow}{rgb}{0.98, 0.98, 0.82}     % #FAFAD2
\definecolor{lightgray}{rgb}{0.83, 0.83, 0.83}                % #D3D3D3
\definecolor{lightgreen}{rgb}{0.56, 0.93, 0.56}               % #90EE90
\definecolor{lightkhaki}{rgb}{0.94, 0.9, 0.55}                % #F0E68C
\definecolor{lightmauve}{rgb}{0.86, 0.82, 1.0}                % #DCD0FF
\definecolor{lightpastelpurple}{rgb}{0.69, 0.61, 0.85}        % #B19CD9
\definecolor{lightpink}{rgb}{1.0, 0.71, 0.76}                 % #FFB6C1
\definecolor{lightsalmon}{rgb}{1.0, 0.63, 0.48}               % #FFA07A
\definecolor{lightsalmonpink}{rgb}{1.0, 0.6, 0.6}             % #FF9999
\definecolor{lightseagreen}{rgb}{0.13, 0.7, 0.67}             % #20B2AA
\definecolor{lightskyblue}{rgb}{0.53, 0.81, 0.98}             % #87CEEB
\definecolor{lightslategray}{rgb}{0.47, 0.53, 0.6}            % #778899
\definecolor{lighttaupe}{rgb}{0.7, 0.55, 0.43}                % #B38B6D
\definecolor{lightthulianpink}{rgb}{0.9, 0.56, 0.67}          % #E68FAC
\definecolor{lightyellow}{rgb}{1.0, 1.0, 0.88}                % #FFFFED
\definecolor{lilac}{rgb}{0.78, 0.64, 0.78}                    % #C8A2C8
\definecolor{lime(colorwheel)}{rgb}{0.75, 1.0, 0.0}           % #BFFF00
\definecolor{lime(web)(x11green)}{rgb}{0.0, 1.0, 0.0}         % #00FF00
\definecolor{limegreen}{rgb}{0.2, 0.8, 0.2}                   % #32CD32
\definecolor{lincolngreen}{rgb}{0.11, 0.35, 0.02}             % #195905
\definecolor{linen}{rgb}{0.98, 0.94, 0.9}                     % #FAF0E6
\definecolor{liver}{rgb}{0.33, 0.29, 0.31}                    % #534B4F
\definecolor{lust}{rgb}{0.9, 0.13, 0.13}                      % #E62020
\definecolor{macaroniandcheese}{rgb}{1.0, 0.74, 0.53}         % #FFBD88
\definecolor{magenta}{rgb}{1.0, 0.0, 1.0}                     % #FF00FF
\definecolor{magenta(dye)}{rgb}{0.79, 0.08, 0.48}             % #CA1F7B
\definecolor{magenta(process)}{rgb}{1.0, 0.0, 0.56}           % #FF0090
\definecolor{magicmint}{rgb}{0.67, 0.94, 0.82}                % #AAF0D1
\definecolor{magnolia}{rgb}{0.97, 0.96, 1.0}                  % #F8F4FF
\definecolor{mahogany}{rgb}{0.75, 0.25, 0.0}                  % #C04000
\definecolor{maize}{rgb}{0.98, 0.93, 0.37}                    % #FBEC5D
\definecolor{majorelleblue}{rgb}{0.38, 0.31, 0.86}            % #6050DC
\definecolor{malachite}{rgb}{0.04, 0.85, 0.32}                % #0BDA51
\definecolor{manatee}{rgb}{0.59, 0.6, 0.67}                   % #979AAA
\definecolor{mangotango}{rgb}{1.0, 0.51, 0.26}                % #FF8243
\definecolor{maroon(html/css)}{rgb}{0.5, 0.0, 0.0}            % #800000
\definecolor{maroon(x11)}{rgb}{0.69, 0.19, 0.38}              % #B03060
\definecolor{mauve}{rgb}{0.88, 0.69, 1.0}                     % #E0B0FF
\definecolor{mauvetaupe}{rgb}{0.57, 0.37, 0.43}               % #915F6D
\definecolor{mauvelous}{rgb}{0.94, 0.6, 0.67}                 % #EF98AA
\definecolor{mayablue}{rgb}{0.45, 0.76, 0.98}                 % #73C2FB
\definecolor{meatbrown}{rgb}{0.9, 0.72, 0.23}                 % #E5B73B
\definecolor{mediumaquamarine}{rgb}{0.4, 0.8, 0.67}           % #66DDAA
\definecolor{mediumblue}{rgb}{0.0, 0.0, 0.8}                  % #0000CD
\definecolor{mediumcandyapplered}{rgb}{0.89, 0.02, 0.17}      % #E2062C
\definecolor{mediumcarmine}{rgb}{0.69, 0.25, 0.21}            % #AF4035
\definecolor{mediumchampagne}{rgb}{0.95, 0.9, 0.67}           % #F3E5AB
\definecolor{mediumelectricblue}{rgb}{0.01, 0.31, 0.59}       % #035096
\definecolor{mediumjunglegreen}{rgb}{0.11, 0.21, 0.18}        % #1C352D
\definecolor{mediumlavendermagenta}{rgb}{0.8, 0.6, 0.8}       % #DDA0DD
\definecolor{mediumorchid}{rgb}{0.73, 0.33, 0.83}             % #BA55D3
\definecolor{mediumpersianblue}{rgb}{0.0, 0.4, 0.65}          % #0067A5
\definecolor{mediumpurple}{rgb}{0.58, 0.44, 0.86}             % #9370DB
\definecolor{mediumred-violet}{rgb}{0.73, 0.2, 0.52}          % #BB3385
\definecolor{mediumseagreen}{rgb}{0.24, 0.7, 0.44}            % #3CB371
\definecolor{mediumslateblue}{rgb}{0.48, 0.41, 0.93}          % #7B68EE
\definecolor{mediumspringbud}{rgb}{0.79, 0.86, 0.54}          % #C9DC87
\definecolor{mediumspringgreen}{rgb}{0.0, 0.98, 0.6}          % #00FA9A
\definecolor{mediumtaupe}{rgb}{0.4, 0.3, 0.28}                % #674C47
\definecolor{mediumtealblue}{rgb}{0.0, 0.33, 0.71}            % #0054B4
\definecolor{mediumturquoise}{rgb}{0.28, 0.82, 0.8}           % #48D1CC
\definecolor{mediumviolet-red}{rgb}{0.78, 0.08, 0.52}         % #C71585
\definecolor{melon}{rgb}{0.99, 0.74, 0.71}                    % #FDBCB4
\definecolor{midnightblue}{rgb}{0.1, 0.1, 0.44}               % #191970
\definecolor{midnightgreen(eaglegreen)}{rgb}{0.0, 0.29, 0.33} % #004953
\definecolor{mikadoyellow}{rgb}{1.0, 0.77, 0.05}              % #FFC40C
\definecolor{mint}{rgb}{0.24, 0.71, 0.54}                     % #3EB489
\definecolor{mintcream}{rgb}{0.96, 1.0, 0.98}                 % #F5FFFA
\definecolor{mintgreen}{rgb}{0.6, 1.0, 0.6}                   % #98FF98
\definecolor{mistyrose}{rgb}{1.0, 0.89, 0.88}                 % #FFE4E1
\definecolor{moccasin}{rgb}{0.98, 0.92, 0.84}                 % #FAEBD7
\definecolor{modebeige}{rgb}{0.59, 0.44, 0.09}                % #967117
\definecolor{moonstoneblue}{rgb}{0.45, 0.66, 0.76}            % #73A9C2
\definecolor{mordantred19}{rgb}{0.68, 0.05, 0.0}              % #AE0C00
\definecolor{mossgreen}{rgb}{0.68, 0.87, 0.68}                % #ADDFAD
\definecolor{mountainmeadow}{rgb}{0.19, 0.73, 0.56}           % #30BA8F
\definecolor{mountbattenpink}{rgb}{0.6, 0.48, 0.55}           % #997A8D
\definecolor{mulberry}{rgb}{0.77, 0.29, 0.55}                 % #C54B8C
\definecolor{mustard}{rgb}{1.0, 0.86, 0.35}                   % #FFDB58
\definecolor{myrtle}{rgb}{0.13, 0.26, 0.12}                   % #21421E
\definecolor{msugreen}{rgb}{0.09, 0.27, 0.23}                 % #18453B
\definecolor{nadeshikopink}{rgb}{0.96, 0.68, 0.78}            % #F6ADC6
\definecolor{napiergreen}{rgb}{0.16, 0.5, 0.0}                % #2A8000
\definecolor{naplesyellow}{rgb}{0.98, 0.85, 0.37}             % #FADA5E
\definecolor{navajowhite}{rgb}{1.0, 0.87, 0.68}               % #FFDEAD
\definecolor{navyblue}{rgb}{0.0, 0.0, 0.5}                    % #000080
\definecolor{neoncarrot}{rgb}{1.0, 0.64, 0.26}                % #FFA343
\definecolor{neonfuchsia}{rgb}{1.0, 0.25, 0.39}               % #FE59C2
\definecolor{neongreen}{rgb}{0.22, 0.88, 0.08}                % #39FF14
\definecolor{non-photoblue}{rgb}{0.64, 0.87, 0.93}            % #A4DDED
\definecolor{oceanboatblue}{rgb}{0.0, 0.47, 0.75}             % #0077BE
\definecolor{ochre}{rgb}{0.8, 0.47, 0.13}                     % #CC7722
\definecolor{officegreen}{rgb}{0.0, 0.5, 0.0}                 % #008000
\definecolor{oldgold}{rgb}{0.81, 0.71, 0.23}                  % #CFB53B
\definecolor{oldlace}{rgb}{0.99, 0.96, 0.9}                   % #FDF5E6
\definecolor{oldlavender}{rgb}{0.47, 0.41, 0.47}              % #796878
\definecolor{oldmauve}{rgb}{0.4, 0.19, 0.28}                  % #673147
\definecolor{oldrose}{rgb}{0.75, 0.5, 0.51}                   % #C08081
\definecolor{olive}{rgb}{0.5, 0.5, 0.0}                       % #808000
\definecolor{olivedrab(web)(olivedrab3)}{rgb}{0.42, 0.56, 0.14}% #6B8E23
\definecolor{olivedrab7}{rgb}{0.24, 0.2, 0.12}                % #3C341F
\definecolor{olivine}{rgb}{0.6, 0.73, 0.45}                   % #9AB973
\definecolor{onyx}{rgb}{0.06, 0.06, 0.06}                     % #0F0F0F
\definecolor{operamauve}{rgb}{0.72, 0.52, 0.65}               % #B784A7
\definecolor{orange(colorwheel)}{rgb}{1.0, 0.5, 0.0}          % #FF7F00
\definecolor{orange(ryb)}{rgb}{0.98, 0.6, 0.01}               % #FB9902
\definecolor{orange(webcolor)}{rgb}{1.0, 0.65, 0.0}           % #FFA500
\definecolor{orangepeel}{rgb}{1.0, 0.62, 0.0}                 % #FF9F00
\definecolor{orange-red}{rgb}{1.0, 0.27, 0.0}                 % #FF4500
\definecolor{orchid}{rgb}{0.85, 0.44, 0.84}                   % #DA70D6
\definecolor{otterbrown}{rgb}{0.4, 0.26, 0.13}                % #654321
\definecolor{outerspace}{rgb}{0.25, 0.29, 0.3}                % #414A4C
\definecolor{outrageousorange}{rgb}{1.0, 0.43, 0.29}          % #FF6E4A
\definecolor{oxfordblue}{rgb}{0.0, 0.13, 0.28}                % #002147
\definecolor{oucrimsonred}{rgb}{0.6, 0.0, 0.0}                % #990000
\definecolor{pakistangreen}{rgb}{0.0, 0.4, 0.0}               % #006600
\definecolor{palatinateblue}{rgb}{0.15, 0.23, 0.89}           % #273BE2
\definecolor{palatinatepurple}{rgb}{0.41, 0.16, 0.38}         % #682860
\definecolor{paleaqua}{rgb}{0.74, 0.83, 0.9}                  % #BCD4E6
\definecolor{paleblue}{rgb}{0.69, 0.93, 0.93}                 % #AFEEEE
\definecolor{palebrown}{rgb}{0.6, 0.46, 0.33}                 % #987654
\definecolor{palecarmine}{rgb}{0.69, 0.25, 0.21}              % #AF4035
\definecolor{palecerulean}{rgb}{0.61, 0.77, 0.89}             % #9BC4E2
\definecolor{palechestnut}{rgb}{0.87, 0.68, 0.69}             % #DDADAF
\definecolor{palecopper}{rgb}{0.85, 0.54, 0.4}                % #DA8A67
\definecolor{palecornflowerblue}{rgb}{0.67, 0.8, 0.94}        % #ABCDEF
\definecolor{palegold}{rgb}{0.9, 0.75, 0.54}                  % #E6BE8A
\definecolor{palegoldenrod}{rgb}{0.93, 0.91, 0.67}            % #EEE8AA
\definecolor{palegreen}{rgb}{0.6, 0.98, 0.6}                  % #98FB98
\definecolor{palemagenta}{rgb}{0.98, 0.52, 0.9}               % #F984E5
\definecolor{palepink}{rgb}{0.98, 0.85, 0.87}                 % #FADADD
\definecolor{paleplum}{rgb}{0.8, 0.6, 0.8}                    % #DDA0DD
\definecolor{palered-violet}{rgb}{0.86, 0.44, 0.58}           % #DB7093
\definecolor{palerobineggblue}{rgb}{0.59, 0.87, 0.82}         % #96DED1
\definecolor{palesilver}{rgb}{0.79, 0.75, 0.73}               % #C9C0BB
\definecolor{palespringbud}{rgb}{0.93, 0.92, 0.74}            % #ECEBBD
\definecolor{paletaupe}{rgb}{0.74, 0.6, 0.49}                 % #BC987E
\definecolor{paleviolet-red}{rgb}{0.86, 0.44, 0.58}           % #DB7093
\definecolor{pansypurple}{rgb}{0.47, 0.09, 0.29}              % #78184A
\definecolor{papayawhip}{rgb}{1.0, 0.94, 0.84}                % #FFEFD5
\definecolor{parisgreen}{rgb}{0.31, 0.78, 0.47}               % #50C878
\definecolor{pastelblue}{rgb}{0.68, 0.78, 0.81}               % #AEC6CF
\definecolor{pastelbrown}{rgb}{0.51, 0.41, 0.33}              % #836953
\definecolor{pastelgray}{rgb}{0.81, 0.81, 0.77}               % #CFCFC4
\definecolor{pastelgreen}{rgb}{0.47, 0.87, 0.47}              % #77DD77
\definecolor{pastelmagenta}{rgb}{0.96, 0.6, 0.76}             % #F49AC2
\definecolor{pastelorange}{rgb}{1.0, 0.7, 0.28}               % #FFB347
\definecolor{pastelpink}{rgb}{1.0, 0.82, 0.86}                % #FFD1DC
\definecolor{pastelpurple}{rgb}{0.7, 0.62, 0.71}              % #B39EB5
\definecolor{pastelred}{rgb}{1.0, 0.41, 0.38}                 % #FF6961
\definecolor{pastelviolet}{rgb}{0.8, 0.6, 0.79}               % #CB99C9
\definecolor{pastelyellow}{rgb}{0.99, 0.99, 0.59}             % #FDFD96
\definecolor{patriarch}{rgb}{0.5, 0.0, 0.5}                   % #800080
\definecolor{peach}{rgb}{1.0, 0.9, 0.71}                      % #FFE5B4
\definecolor{peach-orange}{rgb}{1.0, 0.8, 0.6}                % #FFCC99
\definecolor{peachpuff}{rgb}{1.0, 0.85, 0.73}                 % #FFDAB9
\definecolor{peach-yellow}{rgb}{0.98, 0.87, 0.68}             % #FADFAD
\definecolor{pear}{rgb}{0.82, 0.89, 0.19}                     % #D1E231
\definecolor{pearl}{rgb}{0.94, 0.92, 0.84}                    % #F0EAD6
\definecolor{peridot}{rgb}{0.9, 0.89, 0.0}                    % #E6E200
\definecolor{periwinkle}{rgb}{0.8, 0.8, 1.0}                  % #CCCCFF
\definecolor{persianblue}{rgb}{0.11, 0.22, 0.73}              % #1C39BB
\definecolor{persiangreen}{rgb}{0.0, 0.65, 0.58}              % #00A693
\definecolor{persianindigo}{rgb}{0.2, 0.07, 0.48}             % #32127A
\definecolor{persianorange}{rgb}{0.85, 0.56, 0.35}            % #D99058
\definecolor{peru}{rgb}{0.8, 0.52, 0.25}                      % #CD853F
\definecolor{persianpink}{rgb}{0.97, 0.5, 0.75}               % #F77FBE
\definecolor{persianplum}{rgb}{0.44, 0.11, 0.11}              % #701C1C
\definecolor{persianred}{rgb}{0.8, 0.2, 0.2}                  % #CC3333
\definecolor{persianrose}{rgb}{1.0, 0.16, 0.64}               % #FE28A2
\definecolor{persimmon}{rgb}{0.93, 0.35, 0.0}                 % #EC5800
\definecolor{phlox}{rgb}{0.87, 0.0, 1.0}                      % #DF00FF
\definecolor{phthaloblue}{rgb}{0.0, 0.06, 0.54}               % #000F89
\definecolor{phthalogreen}{rgb}{0.07, 0.21, 0.14}             % #123524
\definecolor{piggypink}{rgb}{0.99, 0.87, 0.9}                 % #FDDDE6
\definecolor{pinegreen}{rgb}{0.0, 0.47, 0.44}                 % #01796F
\definecolor{pink}{rgb}{1.0, 0.75, 0.8}                       % #FFC0CB
\definecolor{pink-orange}{rgb}{1.0, 0.6, 0.4}                 % #FF9966
\definecolor{pinkpearl}{rgb}{0.91, 0.67, 0.81}                % #E7ACCF
\definecolor{pinksherbet}{rgb}{0.97, 0.56, 0.65}              % #F78FA7
\definecolor{pistachio}{rgb}{0.58, 0.77, 0.45}                % #93C572
\definecolor{platinum}{rgb}{0.9, 0.89, 0.89}                  % #E5E4E2
\definecolor{plum(traditional)}{rgb}{0.56, 0.27, 0.52}        % #8E4585
\definecolor{plum(web)}{rgb}{0.8, 0.6, 0.8}                   % #DDA0DD
\definecolor{portlandorange}{rgb}{1.0, 0.35, 0.21}            % #FF5A36
\definecolor{powderblue(web)}{rgb}{0.69, 0.88, 0.9}           % #B0E0E6
\definecolor{princetonorange}{rgb}{1.0, 0.56, 0.0}            % #FF8F00
\definecolor{prune}{rgb}{0.44, 0.11, 0.11}                    % #701C1C
\definecolor{prussianblue}{rgb}{0.0, 0.19, 0.33}              % #003153
\definecolor{psychedelicpurple}{rgb}{0.87, 0.0, 1.0}          % #DF00FF
\definecolor{puce}{rgb}{0.8, 0.53, 0.6}                       % #CC8899
\definecolor{pumpkin}{rgb}{1.0, 0.46, 0.09}                   % #FF7518
\definecolor{purple(html/css)}{rgb}{0.5, 0.0, 0.5}            % #800080
\definecolor{purple(munsell)}{rgb}{0.62, 0.0, 0.77}           % #9F00C5
\definecolor{purple(x11)}{rgb}{0.63, 0.36, 0.94}              % #A020F0
\definecolor{purpleheart}{rgb}{0.41, 0.21, 0.61}              % #69359C
\definecolor{purplemountainmajesty}{rgb}{0.59, 0.47, 0.71}    % #9678B6
\definecolor{purplepizzazz}{rgb}{1.0, 0.31, 0.85}             % #FE4EDA
\definecolor{purpletaupe}{rgb}{0.31, 0.25, 0.3}               % #50404D
\definecolor{radicalred}{rgb}{1.0, 0.21, 0.37}                % #FF355E
\definecolor{raspberry}{rgb}{0.89, 0.04, 0.36}                % #E30B5D
\definecolor{raspberryglace}{rgb}{0.57, 0.37, 0.43}           % #915F6D
\definecolor{raspberrypink}{rgb}{0.89, 0.31, 0.61}            % #E25098
\definecolor{raspberryrose}{rgb}{0.7, 0.27, 0.42}             % #B3446C
\definecolor{rawumber}{rgb}{0.51, 0.4, 0.27}                  % #826644
\definecolor{razzledazzlerose}{rgb}{1.0, 0.2, 0.8}            % #FF33CC
\definecolor{razzmatazz}{rgb}{0.89, 0.15, 0.42}               % #E3256B
\definecolor{red}{rgb}{1.0, 0.0, 0.0}                         % #FF0000
\definecolor{red(munsell)}{rgb}{0.95, 0.0, 0.24}              % #F2003C
\definecolor{red(ncs)}{rgb}{0.77, 0.01, 0.2}                  % #C40233
\definecolor{red(pigment)}{rgb}{0.93, 0.11, 0.14}             % #ED1C24
\definecolor{red(ryb)}{rgb}{1.0, 0.15, 0.07}                  % #FE2712
\definecolor{red-brown}{rgb}{0.65, 0.16, 0.16}                % #A52A2A
\definecolor{red-violet}{rgb}{0.78, 0.08, 0.52}               % #C71585
\definecolor{redwood}{rgb}{0.67, 0.31, 0.32}                  % #AB4E52
\definecolor{regalia}{rgb}{0.32, 0.18, 0.5}                   % #522D80
\definecolor{richblack}{rgb}{0.0, 0.25, 0.25}                 % #004040
\definecolor{richbrilliantlavender}{rgb}{0.95, 0.65, 1.0}     % #F1A7FE
\definecolor{richcarmine}{rgb}{0.84, 0.0, 0.25}               % #D70040
\definecolor{richelectricblue}{rgb}{0.03, 0.57, 0.82}         % #0892D0
\definecolor{richlavender}{rgb}{0.67, 0.38, 0.8}              % #A76BCF
\definecolor{richlilac}{rgb}{0.71, 0.4, 0.82}                 % #B666D2
\definecolor{richmaroon}{rgb}{0.69, 0.19, 0.38}               % #B03060
\definecolor{riflegreen}{rgb}{0.25, 0.28, 0.2}                % #414833
\definecolor{robineggblue}{rgb}{0.0, 0.8, 0.8}                % #00CCCC
\definecolor{rose}{rgb}{1.0, 0.0, 0.5}                        % #FF007F
\definecolor{rosebonbon}{rgb}{0.98, 0.26, 0.62}               % #F9429E
\definecolor{roseebony}{rgb}{0.4, 0.3, 0.28}                  % #674846
\definecolor{rosegold}{rgb}{0.72, 0.43, 0.47}                 % #B76E79
\definecolor{rosemadder}{rgb}{0.89, 0.15, 0.21}               % #E32636
\definecolor{rosepink}{rgb}{1.0, 0.4, 0.8}                    % #FF66CC
\definecolor{rosequartz}{rgb}{0.67, 0.6, 0.66}                % #AA98A9
\definecolor{rosetaupe}{rgb}{0.56, 0.36, 0.36}                % #905D5D
\definecolor{rosevale}{rgb}{0.67, 0.31, 0.32}                 % #AB4E52
\definecolor{rosewood}{rgb}{0.4, 0.0, 0.04}                   % #65000B
\definecolor{rossocorsa}{rgb}{0.83, 0.0, 0.0}                 % #D40000
\definecolor{rosybrown}{rgb}{0.74, 0.56, 0.56}                % #BC8F8F
\definecolor{royalazure}{rgb}{0.0, 0.22, 0.66}                % #0038A8
\definecolor{royalblue(traditional)}{rgb}{0.0, 0.14, 0.4}     % #002366
\definecolor{royalblue(web)}{rgb}{0.25, 0.41, 0.88}           % #4169E1
\definecolor{royalfuchsia}{rgb}{0.79, 0.17, 0.57}             % #CA2C92
\definecolor{royalpurple}{rgb}{0.47, 0.32, 0.66}              % #7851A9
\definecolor{ruby}{rgb}{0.88, 0.07, 0.37}                     % #E0115F
\definecolor{ruddy}{rgb}{1.0, 0.0, 0.16}                      % #FF0028
\definecolor{ruddybrown}{rgb}{0.73, 0.4, 0.16}                % #BB6528
\definecolor{ruddypink}{rgb}{0.88, 0.56, 0.59}                % #E18E96
\definecolor{rufous}{rgb}{0.66, 0.11, 0.03}                   % #A81C07
\definecolor{russet}{rgb}{0.5, 0.27, 0.11}                    % #80461B
\definecolor{rust}{rgb}{0.72, 0.25, 0.05}                     % #B7410E
\definecolor{sacramentostategreen}{rgb}{0.0, 0.34, 0.25}      % #00563F
\definecolor{saddlebrown}{rgb}{0.55, 0.27, 0.07}              % #8B4513
\definecolor{safetyorange(blazeorange)}{rgb}{1.0, 0.4, 0.0}   % #FF6700
\definecolor{saffron}{rgb}{0.96, 0.77, 0.19}                  % #F4C430
\definecolor{st.patrick\'sblue}{rgb}{0.14, 0.16, 0.48}        % #23297A
\definecolor{salmon}{rgb}{1.0, 0.55, 0.41}                    % #FF8C69
\definecolor{salmonpink}{rgb}{1.0, 0.57, 0.64}                % #FF91A4
\definecolor{sand}{rgb}{0.76, 0.7, 0.5}                       % #C2B280
\definecolor{sanddune}{rgb}{0.59, 0.44, 0.09}                 % #967117
\definecolor{sandstorm}{rgb}{0.93, 0.84, 0.25}                % #ECD540
\definecolor{sandybrown}{rgb}{0.96, 0.64, 0.38}               % #F4A460
\definecolor{sandytaupe}{rgb}{0.59, 0.44, 0.09}               % #967117
\definecolor{sangria}{rgb}{0.57, 0.0, 0.04}                   % #92000A
\definecolor{sapgreen}{rgb}{0.31, 0.49, 0.16}                 % #507D2A
\definecolor{sapphire}{rgb}{0.03, 0.15, 0.4}                  % #082567
\definecolor{satinsheengold}{rgb}{0.8, 0.63, 0.21}            % #CBA135
\definecolor{scarlet}{rgb}{1.0, 0.13, 0.0}                    % #FF2000
\definecolor{schoolbusyellow}{rgb}{1.0, 0.85, 0.0}            % #FFD800
\definecolor{screamin\'green}{rgb}{0.46, 1.0, 0.44}           % #76FF7A
\definecolor{seagreen}{rgb}{0.18, 0.55, 0.34}                 % #2E8B57
\definecolor{sealbrown}{rgb}{0.2, 0.08, 0.08}                 % #321414
\definecolor{seashell}{rgb}{1.0, 0.96, 0.93}                  % #FFF5EE
\definecolor{selectiveyellow}{rgb}{1.0, 0.73, 0.0}            % #FFBA00
\definecolor{sepia}{rgb}{0.44, 0.26, 0.08}                    % #704214
\definecolor{shadow}{rgb}{0.54, 0.47, 0.36}                   % #8A795D
\definecolor{shamrockgreen}{rgb}{0.0, 0.62, 0.38}             % #009E60
\definecolor{shockingpink}{rgb}{0.99, 0.06, 0.75}             % #FC0FC0
\definecolor{sienna}{rgb}{0.53, 0.18, 0.09}                   % #882D17
\definecolor{silver}{rgb}{0.75, 0.75, 0.75}                   % #C0C0C0
\definecolor{sinopia}{rgb}{0.8, 0.25, 0.04}                   % #CB410B
\definecolor{skobeloff}{rgb}{0.0, 0.48, 0.45}                 % #007474
\definecolor{skyblue}{rgb}{0.53, 0.81, 0.92}                  % #87CEEB
\definecolor{skymagenta}{rgb}{0.81, 0.44, 0.69}               % #CF71AF
\definecolor{slateblue}{rgb}{0.42, 0.35, 0.8}                 % #6A5ACD
\definecolor{slategray}{rgb}{0.44, 0.5, 0.56}                 % #708090
\definecolor{smalt(darkpowderblue)}{rgb}{0.0, 0.2, 0.6}       % #003399
\definecolor{smokeytopaz}{rgb}{0.58, 0.25, 0.03}              % #933D41
\definecolor{smokyblack}{rgb}{0.06, 0.05, 0.03}               % #100C08
\definecolor{snow}{rgb}{1.0, 0.98, 0.98}                      % #FFFAFA
\definecolor{spirodiscoball}{rgb}{0.06, 0.75, 0.99}           % #0FC0FC
\definecolor{splashedwhite}{rgb}{1.0, 0.99, 1.0}              % #FEFDFF
\definecolor{springbud}{rgb}{0.65, 0.99, 0.0}                 % #A7FC00
\definecolor{springgreen}{rgb}{0.0, 1.0, 0.5}                 % #00FF7F
\definecolor{steelblue}{rgb}{0.27, 0.51, 0.71}                % #4682B4
\definecolor{stildegrainyellow}{rgb}{0.98, 0.85, 0.37}        % #FADA5E
\definecolor{straw}{rgb}{0.89, 0.85, 0.44}                    % #E4D96F
\definecolor{sunglow}{rgb}{1.0, 0.8, 0.2}                     % #FFCC33
\definecolor{sunset}{rgb}{0.98, 0.84, 0.65}                   % #FAD6A5
\definecolor{tan}{rgb}{0.82, 0.71, 0.55}                      % #D2B48C
\definecolor{tangelo}{rgb}{0.98, 0.3, 0.0}                    % #F94D00
\definecolor{tangerine}{rgb}{0.95, 0.52, 0.0}                 % #F28500
\definecolor{tangerineyellow}{rgb}{1.0, 0.8, 0.0}             % #FFCC00
\definecolor{taupe}{rgb}{0.28, 0.24, 0.2}                     % #483C32
\definecolor{taupegray}{rgb}{0.55, 0.52, 0.54}                % #8B8589
\definecolor{teagreen}{rgb}{0.82, 0.94, 0.75}                 % #D0F0C0
\definecolor{tearose(orange)}{rgb}{0.97, 0.51, 0.47}          % #F88379
\definecolor{tearose(rose)}{rgb}{0.96, 0.76, 0.76}            % #F4C2C2
\definecolor{teal}{rgb}{0.0, 0.5, 0.5}                        % #008080
\definecolor{tealblue}{rgb}{0.21, 0.46, 0.53}                 % #367588
\definecolor{tealgreen}{rgb}{0.0, 0.51, 0.5}                  % #006D5B
\definecolor{tenné(tawny)}{rgb}{0.8, 0.34, 0.0}               % #CD5700
\definecolor{terracotta}{rgb}{0.89, 0.45, 0.36}               % #E2725B
\definecolor{thistle}{rgb}{0.85, 0.75, 0.85}                  % #D8BFD8
\definecolor{thulianpink}{rgb}{0.87, 0.44, 0.63}              % #DE6FA1
\definecolor{ticklemepink}{rgb}{0.99, 0.54, 0.67}             % #FC89AC
\definecolor{tiffanyblue}{rgb}{0.04, 0.73, 0.71}              % #0ABAB5
\definecolor{tiger\'seye}{rgb}{0.88, 0.55, 0.24}              % #E08D3C
\definecolor{timberwolf}{rgb}{0.86, 0.84, 0.82}               % #DBD7D2
\definecolor{titaniumyellow}{rgb}{0.93, 0.9, 0.0}             % #EEE600
\definecolor{tomato}{rgb}{1.0, 0.39, 0.28}                    % #FF6347
\definecolor{toolbox}{rgb}{0.45, 0.42, 0.75}                  % #746CC0
\definecolor{tractorred}{rgb}{0.99, 0.05, 0.21}               % #FD0E35
\definecolor{trolleygrey}{rgb}{0.5, 0.5, 0.5}                 % #808080
\definecolor{tropicalrainforest}{rgb}{0.0, 0.46, 0.37}        % #00755E
\definecolor{trueblue}{rgb}{0.0, 0.45, 0.81}                  % #0073CF
\definecolor{tuftsblue}{rgb}{0.28, 0.57, 0.81}                % #417DC1
\definecolor{tumbleweed}{rgb}{0.87, 0.67, 0.53}               % #DEAA88
\definecolor{turkishrose}{rgb}{0.71, 0.45, 0.51}              % #B57281
\definecolor{turquoise}{rgb}{0.19, 0.84, 0.78}                % #30D5C8
\definecolor{turquoiseblue}{rgb}{0.0, 1.0, 0.94}              % #00FFEF
\definecolor{turquoisegreen}{rgb}{0.63, 0.84, 0.71}           % #A0D6B4
\definecolor{tuscanred}{rgb}{0.51, 0.21, 0.21}                % #823535
\definecolor{twilightlavender}{rgb}{0.54, 0.29, 0.42}         % #8A496B
\definecolor{tyrianpurple}{rgb}{0.4, 0.01, 0.24}              % #66023C
\definecolor{uablue}{rgb}{0.0, 0.2, 0.67}                     % #0033AA
\definecolor{uared}{rgb}{0.85, 0.0, 0.3}                      % #D9004C
\definecolor{ube}{rgb}{0.53, 0.47, 0.76}                      % #8878C3
\definecolor{uclablue}{rgb}{0.33, 0.41, 0.58}                 % #536895
\definecolor{uclagold}{rgb}{1.0, 0.7, 0.0}                    % #FFB300
\definecolor{ufogreen}{rgb}{0.24, 0.82, 0.44}                 % #3CD070
\definecolor{ultramarine}{rgb}{0.07, 0.04, 0.56}              % #120A8F
\definecolor{ultramarineblue}{rgb}{0.25, 0.4, 0.96}           % #4166F5
\definecolor{ultrapink}{rgb}{1.0, 0.44, 1.0}                  % #FF6FFF
\definecolor{umber}{rgb}{0.39, 0.32, 0.28}                    % #635147
\definecolor{unitednationsblue}{rgb}{0.36, 0.57, 0.9}         % #5B92E5
\definecolor{unmellowyellow}{rgb}{1.0, 1.0, 0.4}              % #FFFF66
\definecolor{upforestgreen}{rgb}{0.0, 0.27, 0.13}             % #014421
\definecolor{upmaroon}{rgb}{0.48, 0.07, 0.07}                 % #7B1113
\definecolor{upsdellred}{rgb}{0.68, 0.09, 0.13}               % #AE2029
\definecolor{urobilin}{rgb}{0.88, 0.68, 0.13}                 % #E1AD21
\definecolor{usccardinal}{rgb}{0.6, 0.0, 0.0}                 % #990000
\definecolor{uscgold}{rgb}{1.0, 0.8, 0.0}                     % #FFCC00
\definecolor{utahcrimson}{rgb}{0.83, 0.0, 0.25}               % #D3003F
\definecolor{vanilla}{rgb}{0.95, 0.9, 0.67}                   % #F3E5AB
\definecolor{vegasgold}{rgb}{0.77, 0.7, 0.35}                 % #C5B358
\definecolor{venetianred}{rgb}{0.78, 0.03, 0.08}              % #C80815
\definecolor{verdigris}{rgb}{0.26, 0.7, 0.68}                 % #43B3AE
\definecolor{vermilion}{rgb}{0.89, 0.26, 0.2}                 % #E34234
\definecolor{veronica}{rgb}{0.63, 0.36, 0.94}                 % #A020F0
\definecolor{violet}{rgb}{0.56, 0.0, 1.0}                     % #8F00FF
\definecolor{violet(colorwheel)}{rgb}{0.5, 0.0, 1.0}          % #7F00FF
\definecolor{violet(ryb)}{rgb}{0.53, 0.0, 0.69}               % #8601AF
\definecolor{violet(web)}{rgb}{0.93, 0.51, 0.93}              % #EE82EE
\definecolor{viridian}{rgb}{0.25, 0.51, 0.43}                 % #40826D
\definecolor{vividauburn}{rgb}{0.58, 0.15, 0.14}              % #922724
\definecolor{vividburgundy}{rgb}{0.62, 0.11, 0.21}            % #9F1D35
\definecolor{vividcerise}{rgb}{0.85, 0.11, 0.51}              % #DA1D81
\definecolor{vividtangerine}{rgb}{1.0, 0.63, 0.54}            % #FFA089
\definecolor{vividviolet}{rgb}{0.62, 0.0, 1.0}                % #9F00FF
\definecolor{warmblack}{rgb}{0.0, 0.26, 0.26}                 % #004242
\definecolor{wenge}{rgb}{0.39, 0.33, 0.32}                    % #645452
\definecolor{wheat}{rgb}{0.96, 0.87, 0.7}                     % #F5DEB3
\definecolor{white}{rgb}{1.0, 1.0, 1.0}                       % #FFFFFF
\definecolor{whitesmoke}{rgb}{0.96, 0.96, 0.96}               % #F5F5F5
\definecolor{wildblueyonder}{rgb}{0.64, 0.68, 0.82}           % #A2ADD0
\definecolor{wildstrawberry}{rgb}{1.0, 0.26, 0.64}            % #FF43A4
\definecolor{wildwatermelon}{rgb}{0.99, 0.42, 0.52}           % #FC6C85
\definecolor{wisteria}{rgb}{0.79, 0.63, 0.86}                 % #C9A0DC
\definecolor{xanadu}{rgb}{0.45, 0.53, 0.47}                   % #738678
\definecolor{yaleblue}{rgb}{0.06, 0.3, 0.57}                  % #0F4D92
\definecolor{yellow}{rgb}{1.0, 1.0, 0.0}                      % #FFFF00
\definecolor{yellow(munsell)}{rgb}{0.94, 0.8, 0.0}            % #EFCC00
\definecolor{yellow(ncs)}{rgb}{1.0, 0.83, 0.0}                % #FFD300
\definecolor{yellow(process)}{rgb}{1.0, 0.94, 0.0}            % #FFEF00
\definecolor{yellow(ryb)}{rgb}{1.0, 1.0, 0.2}                 % #FEFE33
\definecolor{yellow-green}{rgb}{0.6, 0.8, 0.2}                % #9ACD32
\definecolor{zaffre}{rgb}{0.0, 0.08, 0.66}                    % #0014A8
\definecolor{zinnwalditebrown}{rgb}{0.17, 0.09, 0.03}         % #2C1608

\colorlet{refcolor}{green(html/cssgreen)}
\colorlet{journal_color}{hanblue!80!black}
\colorlet{assumption_color}{bronze!80!black}
\colorlet{contrast_color_1}{shockingpink!60!black}
\colorlet{contrast_color_2}{neongreen!40!black}
\colorlet{contrast_color_3}{cyan!60!black}
\colorlet{contrast_color_1_bg}{shockingpink!90!black}
\colorlet{contrast_color_2_bg}{neongreen!90!black}
\colorlet{contrast_color_3_bg}{cyan!90!black}
\colorlet{grayout}{gray!80!black}

\usepackage{makecell}
\usepackage{tabstackengine}
\renewcommand\theadalign{bc}
\renewcommand\theadfont{\bfseries}
\renewcommand\theadgape{\Gape[4pt]}
\renewcommand\cellgape{\Gape[4pt]}
%
%
% % The following is to increase the gap in underbrace.
% % Followed by https://tex.stackexchange.com/questions/13843/vertical-spacing-with-underbrace-command/13864
% \newcommand*\mystrut[1]{\vrule width0pt height0pt depth#1\relax}

% % % Define danger sign


\mode<presentation>
{
%      \usetheme{Warsaw}
%     \usetheme{JuanLesPins}
%  \usetheme{Hannover}
%  \usetheme{Montpellier}
   % \useoutertheme{default}
  % or ...

  \setbeamercovered{transparent}
  % or whatever (possibly just delete it)
 \setbeamertemplate{frametitle}{
  \begin{centering}
    \color{lgtblue!70!black}{\insertframetitle}
    \par
  \end{centering}
  }
}
% \usefoottemplate{\hfill \insertframenumber{} / 27}

% Remove the negation symbols
\setbeamertemplate{navigation symbols}{} % remove navigation symbols

\def\shifpagenumber{9}
\setbeamertemplate{footline}{%
  \hfill%
  \insertframenumber\,/\,\the\numexpr\inserttotalframenumber-\shifpagenumber\relax\, {(+ \shifpagenumber)}%
  \hspace*{0.3cm}%
  \vspace{0.07cm}%
}
% \inserttotalframenumber

\usepackage[english]{babel}
% or whatever

% \usepackage[latin1]{inputenc}
% or whatever

\usepackage{times}
\usepackage[T1]{fontenc}
% Or whatever. Note that the encoding and the font should match. If T1
% does not look nice, try deleting the line with the fontenc.


\usepackage{pgfpages}
% \setbeameroption{show notes}
% \setbeamertemplate{note page}[plain]
% \setbeameroption{second mode text on second screen=right}
% \setbeameroption{show notes on second screen=right}
%

\AtBeginSection[]
  {
     \begin{frame}<beamer>
     \frametitle{Plan}
     \tableofcontents[currentsection]
     \end{frame}
  }

% -------------------------------------------------
% Biber for bibliography
% -------------------------------------------------
% \usepackage[backend=biber,
%           style=alphabetic,
%           natbib=true,
%           abbreviate=true,
%           maxbibnames=99,
%           ]{biblatex}
\usepackage[strict=true,style=english]{csquotes}
\usepackage[backend=biber,
            style=apa,
            % style=numeric,
            natbib=true,
            maxbibnames=99,
            maxnames=99,
            sorting=nyt,
            abbreviate=true]{biblatex}
\AtEveryBibitem{% Clean up the bibtex rather than editing it
 \clearfield{doi}
 \clearfield{url}
 \clearfield{issn}
 \clearlist{location}
 \clearfield{month}
 \clearfield{series}

 \ifentrytype{book}{}{% Remove publisher and editor except for books
  \clearlist{publisher}
  \clearname{editor}
 }
}
\addbibresource{All.bib}
% \addbibresource{Talk-LeChen-FSU_1024_White.bib}

\newcommand{\mycite}[1]{\textcolor{green!80!black}{\small\cite{#1}}}

\def\polhk#1{\setbox0=\hbox{#1}{\ooalign{\hidewidth
  \lower1.5ex\hbox{`}\hidewidth\crcr\unhbox0}}} \def\cprime{$'$}

% The following is to make groups of references.
\newcommand\RefGroup[2]{
  \newrefsection
  \noindent{\Large #1}:\\[1em] #2
  \begingroup
  \renewcommand*{\bibfont}{\small}
  \setlength\bibitemsep{0.7em}
  \printbibliography
  \endgroup
}
\newcommand\RefSmallGroup[2]{
  \newrefsection
  \noindent{\small #1}:\\[0.2em] #2
  \begingroup
  \renewcommand*{\bibfont}{\footnotesize}
  \setlength\bibitemsep{0.1em}
  \printbibliography
  \endgroup
}

% My abbreviate cite in slices.
\DeclareNameFormat{authorcolor}{%
  \textcolor{green}{\namepartfamily}%
  \addcomma\addspace\textcolor{green}{\namepartgiveni}%
  \ifthenelse{\value{listcount}<\value{liststop}}
    {\addcomma\space}{}
}
% Adjust customcite to use the colored author name format
\DeclareCiteCommand{\BriefCite}
  {\usebibmacro{prenote}}
  {\printnames[authorcolor]{author}%
   \setunit{\printdelim{nametitledelim}}%
   \printfield{journaltitle}%
   \setunit{\printdelim{nametitledelim}}%
   \printfield{year}}
  {\multicitedelim}
  {\usebibmacro{postnote}}

\newcommand{\myFootCite}[1]{
  \vfill
  \myfootnoteline \vspace{-1em}
  {\footnotesize \BriefCite{#1}}
}

\newcommand\BackgroundShowImage[2]{
  \usebackgroundtemplate{
    \includegraphics[width=\paperwidth,height=\paperheight]{#1}
  }
  \begin{frame}\frametitle{#2}\end{frame}
  \usebackgroundtemplate{}
}

  \setbeamertemplate{bibliography item}[text]
% \setbeamertemplate{bibliography item}{*}

\setbeamercolor{bibliography entry author}{fg=refcolor}
% \setbeamercolor{bibliography entry title}{fg=black}
% \setbeamercolor{bibliography entry title}{fg=black}
\setbeamercolor{bibliography entry location}{fg=journal_color}
\setbeamercolor{bibliography entry note}{fg=journal_color}
%
\setbeamerfont{bibliography entry author}{size=\footnotesize}
\setbeamerfont{bibliography entry title}{size=\footnotesize}
\setbeamerfont{bibliography entry location}{size=\footnotesize}
\setbeamerfont{bibliography entry note}{size=\footnotesize}


% \makeatletter
% \patchcmd{\beamer@sectionintoc}{\vskip1.5em}{\vskip2.5em}{}{}
% \patchcmd{\beamer@sectionintoc}{\vfill}{\vskip2.2em}{}{}
% \makeatother

\setbeamertemplate{subsection in toc}{%
\leavevmode\leftskip=3.65ex%
  % \llap{\raisebox{0.2ex}{\textcolor{black}{$\blacktriangleright$}}\kern1ex}%
  % \llap{a}%
  \textcolor{black}{\inserttocsubsection}\par%
}
